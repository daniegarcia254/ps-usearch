\documentclass[10pt,spanish]{article}

%----------------------------------------------------------------------------------------
% Codificación, y usar una fuente similar a la Palatino (y no Latin Modern Roman)
\usepackage[utf8]{inputenc} % Acentos, etc.
\usepackage[spanish]{babel} % Castellano
\usepackage[T1]{fontenc}
\renewcommand{\rmdefault}{phv} % Arial (se parece lo suficiente)
\renewcommand{\sfdefault}{phv} % Arial (se parece lo suficiente)

%----------------------------------------------------------------------------------------
% Margenes
\usepackage{geometry}
\geometry{
    a4paper,
    total={210mm,297mm},
    left=30mm,
    right=30mm,
    top=25mm,
    bottom=25mm,
    headheight=15mm
}
%----------------------------------------------------------------------------------------
% Modificar tanto el estilo de las secciones como del indice
\usepackage{titlesec,titletoc}
\titleformat{\section}{\bfseries\Large\uppercase}{\thesection .}{0.8em}{}
\titleformat{\subsection}{\bfseries\large}{\thesubsection .}{0.8em}{}
\titleformat{\subsubsection}{\bfseries}{\thesubsubsection .}{0.8em}{}

% Secciones en una nueva pag
\let\stdsection\section
\renewcommand\section{\newpage\stdsection} 

%----------------------------------------------------------------------------------------
% Colores
\usepackage{xcolor,colortbl} % Colores personalizados y color fondo tablas
\definecolor{grisCabeceraTabla}{RGB}{220,220,220}
\definecolor{grisHeader}{RGB}{180,180,180}

%----------------------------------------------------------------------------------------
% Imagenes
\usepackage{graphicx}
\DeclareGraphicsExtensions{.svg,.eps,.ps,.pdf,.png,.jpg,.jpeg}
\addto{\captionsspanish}{\renewcommand{\listfigurename}{\bfseries\Large Figuras}}
% \usepackage{float}

% COMO MOSTRAR Y REFERENCIAR IMAGENES
% \begin{figure}[h!]
%    \centering
%    \includegraphics[width=0.95\textwidth]{img/nombreImg} % Expecificar tam y que imagen mostrar
%    \caption{Texto explicativo que aparece debajo de la imagen}
%    \label{fig:nombreRefFigura} % nombre con el que referenciar luego, asi \cref{fig:nombreRefFigura}.
% \end{figure}

% QUE OPCIONES USAR EN IMAGENES
% h	 Place the float here, i.e., approximately at the same point it occurs in the source 
%    text (however, not exactly at the spot)
% t	 Position at the top of the page.
% b  Position at the bottom of the page.
% p	 Put on a special page for floats only.
% !	 Override internal parameters LaTeX uses for determining "good" float positions.
% H	 Places the float at precisely the location in the LaTeX code. Requires the float package
%    [1] e.g., \usepackage{float}. This is somewhat equivalent to h!.

%----------------------------------------------------------------------------------------
% FANCY HEADER
\usepackage{fancyhdr}
\fancyhf{} % borrar todos los ajustes

% Configurar la cabecera: fancyhead, fancyfoot para el pie.
\fancyhead[R]{\nouppercase 30243-Proyecto Software 13/14 – Equipo {\bf 20}}
\fancyfoot[C]{\thepage}
\renewcommand{\headrulewidth}{0pt}    % Tam. separador cabecera
\renewcommand{\footrulewidth}{0pt}    % Tam. separador pie
\renewcommand{\headwidth}{6in}        % Anchura cabecera 

%----------------------------------------------------------------------------------------
% Curriculum Vitae
\usepackage{pdfpages}
\usepackage{currvita} 
\setlength{\cvlabelwidth}{40mm}

%----------------------------------------------------------------------------------------
% Tablas (entre otras, las actas de las reuniones)
\linespread {1.4} % Tamaño que ocupa una línea (celda)

\usepackage{array}
\usepackage{longtable}
\setlength\LTleft{0pt}
\setlength\LTright{0pt}
\setlength{\columnsep}{0em}

% Definir los estilos de las celdas para indentar el texto
\newcolumntype{L}[1]{>{\raggedright\let\newline\\\arraybackslash\hspace{0pt}}m{#1}}
\newcolumntype{C}[1]{>{\centering\let\newline\\\arraybackslash\hspace{0pt}}m{#1}}
\newcolumntype{R}[1]{>{\raggedleft\let\newline\\\arraybackslash\hspace{0pt}}m{#1}}

% Filas con columnas multiples
\newcommand{\mc}[2]{\multicolumn{#1}{|C{\dimexpr 1\linewidth-2\tabcolsep}|}{#2}}

% Símbolo del euro
\usepackage{eurosym}

% Colores de lineas de las tablas
\arrayrulecolor{grisHeader}

% Eliminar espacio previo en lista de ítems
\usepackage{enumitem}
\setlist{nolistsep}

% Comandos especiales para crear items dentro de tabla y poder hacer
% una tabla de multiples paginas y que se pueda realizar el salto de
% pagina en cualquier punto.
\usepackage{scrextend}
\newcommand{\cabeceraTabla}[1]{\rowcolor{grisCabeceraTabla}{\bf #1}}
\newcommand{\espacioSubtabla}{\\[0.5ex]}

\newenvironment{lvlOneItem}
	{\begin{addmargin}[2.5em]{1em}
	\vspace{-1em}
	\hspace{-1em}$\bullet$\hspace{0.5em}}
	{\vspace{-2em}\end{addmargin}}
\newenvironment{lvlTwoItem}
	{\begin{addmargin}[4em]{1em}
	\vspace{-1em}
	\hspace{-1em}$\circ$\hspace{0.5em}}
	{\vspace{-2em}\end{addmargin}}
\newenvironment{lvlThreeItem}
	{\begin{addmargin}[5.5em]{1em}
	\vspace{-1em}
	\hspace{-1em}$\blacktriangleright$\hspace{0.5em}}
	{\vspace{-2em}\end{addmargin}}

\newcommand{\itemNvlUno}[1]{\begin{lvlOneItem}{#1}\end{lvlOneItem}\\}
\newcommand{\itemNvlDos}[1]{\begin{lvlTwoItem}{#1}\end{lvlTwoItem}\\}
\newcommand{\itemNvlTres}[1]{\begin{lvlThreeItem}{#1}\end{lvlThreeItem}\\}

%----------------------------------------------------------------------------------------
% Definir el titulo
\newcommand{\nombreDelProyecto}{$\mu Search$}


\newcommand{\singlelinebreak}{\\[\baselineskip]}
\newcommand{\multiplelinebreak}[1]{\\[#1\baselineskip]}
\newlength{\drop}

\newcommand*{\titulo}{\begingroup
\thispagestyle{empty}
\drop = 0.13\textheight
\centering
\vfill
\vspace*{\drop}
\includegraphics[scale=0.25]{img/bitparty_big}\singlelinebreak
{\Huge\bf BITPARTY}\multiplelinebreak{2}
{\huge Proyecto \nombreDelProyecto}\multiplelinebreak{2}
{\Large Versión  1.0}\multiplelinebreak{1}
{\Large 06/2014}\multiplelinebreak{1}
{\em Alberto Berbel Aznar}\\
{\em Javier Briz Alastrué}\\
{\em Héctor Francia Molinero}\\
{\em Daniel García Páez}\\
{\em Alejandro Gracia Mateo}\\
{\em Simón Ortego Parra}\multiplelinebreak{6}
\includegraphics[scale=0.09]{img/diis_uz_big}\singlelinebreak
\vfill
\vspace*{\drop}
\endgroup}

%----------------------------------------------------------------------------------------
% Metadatos y pdf clickeable
\usepackage{hyperref}
\usepackage{hyperxmp}

\hypersetup{
	pdfauthor={Alberto Berbel Aznar, 
				Javier Briz Alastrué, 
				Héctor Francia Molinero, 
				Daniel García Páez,
				Alejandro Gracia Mateo,
				Simón Ortego Parra},
	pdftitle={E20 - Propuesta del proyecto},
	pdfsubject={Proyecto Software. Grado Ing. Informática. EINA. Unizar},
	pdfkeywords={},
	pdfcopyright={Copyright \copyright\ 2014 
				Alberto Berbel Aznar, 
				Javier Briz Alastrué, 
				Héctor Francia Molinero, 
				Daniel García Páez,
				Alejandro Gracia Mateo,
				Simón Ortego Parra. All rights reserved.},
	pdfproducer={PDFLatex},
	pdfcreator={ps2pdf},
	colorlinks=true,
	linkcolor=black,        % color of internal links (change box color with linkbordercolor)
    citecolor=green,        % color of links to bibliography
    filecolor=cyan,         % color of file links
    urlcolor=cyan           % color of external links
}

% Redefinir el nombre de referencias a lo largo del documento
\usepackage[capitalise]{cleveref} % IMPORTANT: load after hyperref package
\crefdefaultlabelformat{\textbf{#2#1#3}} % boldface only the number

\crefname{figure}{figura}{figuras}
\Crefname{figure}{Figura}{Figuras}
\crefname{listing}{algoritmo}{algoritmos}
\Crefname{listing}{Algoritmo}{Algoritmos}
\crefname{section}{sección}{secciones}
\Crefname{section}{Sección}{Secciones}

% Varios comandos a links relativos a otros ficheros (xls, pdf's, etc.)
\newcommand{\hojaesfuerzos}[2]{\href{run:../../recopilacion_esfuerzos/#1.xls}{#2}}
\newcommand{\informepruebas}[2]{\href{run:../../pruebas_de_sistema/#1.xls}{#2}}
\newcommand{\riesgos}[2]{\href{run:../../riesgos_y_lecciones/#1.xls}{#2}}
\newcommand{\manualUsuario}[2]{\href{run:../../manual_usuario/documento/#1.pdf}{#2}}
\newcommand{\reuniones}[2]{\href{run:../../reuniones/#1.pdf}{#2}}
\newcommand{\presupuesto}[2]{\href{run:../../propuesta_de_proyecto/presupuesto/#1.pdf}{#2}}
\newcommand{\presupuestos}[2]{\href{run:../../propuesta_de_proyecto/presupuesto/#1.xls}{#2}}


%=-=-=-=-=-=-=-=-=-=-=-=-=-=-=-=-=-=-=-=-=-=-=-=-=-=-=-=-=-=-=-=-=-=-=-=-=-=-=-=-=-=-=-=-=-=-=%
%        I N I C I O    D  E  L      D O C U M E N T O            %
%=-=-=-=-=-=-=-=-=-=-=-=-=-=-=-=-=-=-=-=-=-=-=-=-=-=-=-=-=-=-=-=-=-=-=-=-=-=-=-=-=-=-=-=-=-=-=%

% Comenzar sin numeracion en las paginas y sin cabeceras (de momento)
\renewcommand{\thepage}{}
\pagestyle{empty}

%==============================================================================================
% PORTADA
%==============================================================================================

\begin{document}
\titulo
\clearpage

%==============================================================================================
% ÍNDICE
%==============================================================================================	
\pagestyle{fancy}
\tableofcontents

%==============================================================================================
% 01.   INTRODUCCIÓN
% 01.1. IDENTIFICACIÓN DEL PROYECTO
% 01.2. OBJETIVO Y ALCANCE DEL PROYECTO
% 01.3. IDENTIFICACIÓN DEL EQUIPO QUE REALIZA EL PROYECTO
% 01.4. BREVE DESCRIPCION DEL CONTENIDO DEL RESTO DE SECCIONES DE LA MEMORIA, INCLUYENDO ANEXOS
%==============================================================================================

% Continuar con numeracion romana (1,2,3...) 
\renewcommand{\thepage}{\arabic{page}}

% 01
\section{Introducción}
\paragraph{}Este documento trata de guiar al usuario a través de la aplicación web $\mu$Search (un catálogo electronico para la venta de microcontroladores) de forma que este conozca su funcionamiento al detalle y aprenda a navegar a través de sus distintas funcionalidades.

\paragraph{}La guía está dirigida tanto para el usuario o cliente habitual de la aplicación web, que la usará para realizar sus pedidos, como para el administrador de la aplicación web; ya que son dos tipos de usuarios que poseen funcionalidades distintas.

% 01.1
\subsection{Identificación del proyecto}
% 01.1. IDENTIFICACION DEL PROYECTO
%----------------------------------------------------------------------------------------------
\noindent Identificamos el proyecto y su funcionalidad con el siguiente título:
\paragraph{}\textbf{\large \indent Página web para la venta de microcontroladores a través de un \newline \indent catálogo electrónico.}

\noindent \newline Además del título, se ofrece la posibilidad de identificar y referenciar el proyecto a través del siguiente código:
\begin{center}
        \textbf{\textit{\large \\ ps-14-e20-usearch}}
\end{center}
Donde:
        \begin{itemize}
        \renewcommand{\labelitemi}{$-$}
                \item ``ps'' hace referencia a ``Proyecto Software''
                \item ``14'' hace referencia al año 2014
                \item ``e20'' hace a referencia al equipo del proyecto, ``Equipo 20''
                \item ``usearch'' hace referencia al nombre del proyecto, ``\nombreDelProyecto''
        \end{itemize}

% 01.1
\subsection{Objetivo y alcanze}
% 01.2. OBJETIVO Y ALCANCE DEL PROYECTO
%----------------------------------------------------------------------------------------------
La presente aplicación tiene como objetivo dar solución al problema planteado por la empresa cliente.
El cliente nos ha transmitido la necesidad de diseñar un catálogo electrónico de los microconroladores que distribuye, que permita la gestión de los productos disponibles, y que permita a los clientes realizar búsquedas y efectuar pedidos.

% 01.1
\subsection{Identificación del equipo}
% 01.3. IDENTIFICACIÓN DEL EQUIPO QUE REALIZA EL PROYECTO
%----------------------------------------------------------------------------------------------
El proyecto será llevado a cabo por un equipo de 6 personas, compuesto por:
\begin{itemize}
\item Alberto Berbel Aznar
\item Javier Briz Alastrué
\item Héctor Francia Molinero
\item Daniel García Páez
\item Alejandro Gracia Mateo
\item Simón Ortego Parra
\end{itemize}

\paragraph{} El equipo cuenta con experiencia en:
\begin{itemize}

\item El mantenimiento de administración de sistemas.
\item El desarrollo y mantenimiento de sistemas web utilizando diferentes lenguajes como HTML, PHP, CSS, JavaScript.
\item La administración de gestores de contenidos como Drupal, Wordpress... 
\item El desarrollo y mantenimiento de bases de datos.
\item Montar web con funcionalidades similares con gestión de contenido como películas o series.
\item El ámbito de la web.
\item Entornos cliente-servidor.
\item Creación de interfaces web.

\end{itemize}

\paragraph{} Se puede encontrar un mayor detalle sobre la formación de cada miembro del equipo en el anexo \ref{sec:cvs} de Curriculum Vitae.

% 01.4
\subsection{Descripción del contenido}
% 01.4. BREVE DESCRIPCION DEL CONTENIDO DEL RESTO DE SECCIONES DE LA MEMORIA, INCLUYENDO ANEXOS
%----------------------------------------------------------------------------------------------
En la presente memoria se detallan todos los aspectos referentes a la aplicación y al proceso de su desarrollo.

Se detallan los requisitos que el cliente plateó inicialmente para el sistema, la descripción técnica de la solución adoptada, lsa pruebas que se han realizado para verificar que los requisitos se cumplen, los manuales generados, tanto el manual técnico como el de usuario.

También se detallan en esta memoria todas las actividades llevadas a cabo en el proyecto, incluidas las reuniones y sus actas y otra documentación generada en ellas, así como las configuraciones que se han utilizado en el proyecto.

Finalmente se explican las prácticas llevadas a cabo con el fin de asegurar la calidad de la aplicación y se concluye este documento comentando las lecciones aprendidas en el transcurso del proyecto y las conclusiones extraidas del mismo.

Se anexan, con la intención de facilitar la lectura de la memoria, los curriculum vitae de los miembros del equipo, las hojas de esfuerzo de cada uno de ellos, y otros documentos que aportan información adicional a secciones anteriormente mencionadas.

%==============================================================================================
% 02. REQUISITOS DEL SISTEMA
%==============================================================================================

% 02
\section{Requisitos del sistema}
% 02. REQUISITOS DEL SISTEMA
%==============================================================================================
\paragraph{}Se presentan en este apartado los requisitos funcionales con los que el sistema debe cumplir:

% Configuración margenes lista de requisitos
\setlist[description]{leftmargin=\parindent,labelindent=\parindent}
\setlist[enumerate]{leftmargin=0.7in,labelindent=\parindent}
\setlist[itemize]{leftmargin=0.7in,labelindent=\parindent}

\begin{description}
\item[RF-1] Un microcontrolador (elemento) estará compuesto de los siguientes campos: 
	\begin{enumerate}
		\item Referencia (será única para cada elemento).
        \item Arquitectura
        \item Frecuencia (MHz)
        \item Flash (KB)
        \item RAM (KB)
        \item Precio (Euros)
	\end{enumerate}
	
\item[RF-2] Insertar un elemento en el carro de compra.
   
\item[RF-3] Eliminar un elemento del carro de compra.
	
\item[RF-4] Modificar un elemento del carro de compra. Por modificar se entiende alterar el número de unidades de los elementos.

\item[RF-5] Se podrá generar en cualquier momento un listado de todos los elementos del catálogo.

\item[RF-6] Se podrá actualizar varios elementos del carro de manera simultánea. Por actualizar se entiende a recalcular los precios de cada artículo en el caso de que éstos hayan sido modificados.

\item[RF-7] Se podrá en cualquier momento realizar búsquedas de productos en base a un único campo de búsqueda (una y sólo una de las características de un elemento).
	
\item[RF-8] Los resultados de la búsqueda se presentarán como un listado (sin paginación) que mostrará, de cada elemento, todos sus campos en columnas.

\item[RF-9] Los listados de elementos del catálogo estarán ordenados en base al campo arquitectura del elemento.

\item[RF-10] Se permitirá realizar pedidos. Cada vez que se realice un pedido se le pedirá al cliente la introducción de sus datos personales. Es decir, no existirá persistencia de los datos del cliente tras realizar pedidos.

\item[RF-11] Los pedidos contendrán la suficiente información para identificar a los clientes. Además, no permitirán la reserva de los productos solicitados, únicamente generarán un presupuesto con el coste de los productos elegidos.

\item[RF-12] Los datos solicitados del cliente para los pedidos serán los siguientes:
	\begin{enumerate}
		\item Nombre 
		\item Apellidos
	    \item Dirección
	    \item Ciudad
	    \item Provincia
	    \item País
	    \item Código postal
	    \item Teléfono
	    \item Correo electrónico
	    \item CIF y Empresa aparecerán como campos opcionales que servirán de distinción entre particulares y entidades.
	\end{enumerate}

\item[RF-13] Se contará con una vista diferente para la administración del catálogo. Pudiendo un administrador de la empresa realizar las siguientes acciones:
	\begin{itemize}
    	\item Insertar un nuevo elemento en el catálogo.
		\item Eliminar un elemento del catálogo.
        \item Modificar un elemento del catálogo (cambiar cualquiera de sus características).
        \item Generar en cualquier momento un listado completo de todos los elementos del catálogo.
        \item Realizar en cualquier momento búsquedas de productos en base a un único campo de búsqueda (una y sólo una de las características de un elemento).
    \end{itemize}   
\end{description}


%==============================================================================================
% 03.   DESCRIPCIÓN TÉCNICA
% 03.1. ASPECTOS ARQUITECTURALES Y TECNOLOGICOS
% 03.2. MODELOS DE DATOS
% 03.3. INFORMACIÓN DE DISEÑO DE COMPONENTES RELEVANTES DEL SISTEMA
%==============================================================================================

% 03
\section{Descripción técnica}
% 03. DESCRIPCIÓN TÉCNICA
%==============================================================================================
\paragraph{}La solución técnica que se ha dado con el desarrollo de este proyecto está basada en tecnologías web, capaces de resolver tanto los requisitos de interacción de la aplicación con el usuario como los problemas relacionados con tratamiento y persistencia interna de la información.
Concretamente, se utilizará una interfaz web compatible con las últimas versiones de los navegadores más utilizados (más adelante se detallará esto), y se utilizará el framework CodeIgniter, en lenguaje PHP, como base del proyecto. Para el almacenamiento de la información se utilizará una base de datos MySQL.
Para evitar problemas de latencias con la base de datos, y dado el reducido tamaño del sistema, se optará por alojar la base de datos y todo el resto del sistema (servidor web e intérprete PHP) en un mismo servidor.

\noindent Las tecnologías, lenguajes y aplicaciones utilizadas en el desarrollo del proyecto han sido:

\begin{itemize}
\item HTML 5
\item CSS 3
\item PHP 5
\item CodeIgniter 2.1.4
\item MySQL 5.5
\end{itemize}

\noindent Se ha asegurado que la web renderice de forma correcta en los siguientes navegadores:

\begin{itemize}
\item Google Chrome >=30
\item Internet Explorer >=10
\item Mozilla Firefox >=27
\item Opera >=12
\end{itemize}

\noindent La documentación y manuales de usuario se entregaran al cliente en formato PDF.



% 03.1
\subsection{Aspectos arquitecturales y tecnológicos}
% 03.1. ASPECTOS ARQUITECTURALES Y TECNOLOGICOS
%----------------------------------------------------------------------------------------------
El patrón que utilizamos para el diseño arquitectural de nuestro catálogo electrónico es el de Modelo-Vista-Controlador, en concreto, la variante Modelo-Vista-Presentador del mismo (véase la~\cref{fig:diagDespliegue}).

\vspace{.2cm}
\begin{figure}[ht]
	\centerline{\includegraphics[scale=0.75]{img/diagrama_despliegue}}\
	\caption{Diagrama de despliegue del sistema}
	\label{fig:diagDespliegue}
\end{figure}
\paragraph{}
\paragraph{}
\noindent Los componentes de este diseño arquitectural son:

\paragraph{Interfaz web.} Es el componente que representa la vista de nuestra aplicación. Es el componente con el que los usuarios interactuan directamente para navegar por la aplicación y visualizar los resultados que producen las interacciones que realicen con la vista. Las acciones del usuario que impliquen el acceso a los datos del modeo o la modificación/eliminación de los mismos son delegadas al componente Controlador.

\paragraph{Controlador.} Es nuestro componente Presentador, tiene toda la lógica de la vista 
y es responsable de sincronizar el modelo y la vista. Es decir, cuando la vista notifica al 
Presentador que el usuario ha realizado alguna acción (por ejemplo, hacer clic en un botón) que afecta de alguna manera al modelo de datos del sistema, entonces el presentador se encarga de actualizar los datos pertinentes y sincronizar los cambios entre el modelo y la vista.

\paragraph{Base de datos.} Es nuestro componente modelo, se encarga de encapsular 
los datos y ofrecer operaciones para su acceso y procesamiento, es decir, ofrece persistencia de datos para la aplicación. Sólo el componente
Controlador interactúa con el modelo de datos..\newline\newline
Las tecnologías utilizadas son las que se detallaron en el apartado anterior.

% 03.2
\subsection{Modelos de datos}
% 03.2. MODELOS DE DATOS
%----------------------------------------------------------------------------------------------
\paragraph{}Nuestro modelo de datos es muy sencillo puesto que solo necesitamos almacenar en la base de datos la información relacionada con los microcontroladores. El resto de la información que manejamos no se almacena en nuestra base de datos; los microcontroladores que introduce un usuario en el carrito de  la compra, los guardamos temporalmente utilizando las funciones de PHP; y la información asociada a un  cliente que realiza una compra, no se almacena en ningún sitio puesto que solo queda reflejada en la factura que se genera cuando se solicita un pedido. \newline

\noindent Por lo tanto, nuestro modelo de datos es el siguiente:

\begin{figure}[h!]
\centering
\includegraphics[width=0.40\textwidth]{img/modelo_datos}
\caption{Modelo de datos}
 \label{fig:modelo_datos}
\end{figure}


% 03.3
\subsection{Diseño de los componentes}
% 03.3. INFORMACIÓN DE DISEÑO DE COMPONENTES RELEVANTES DEL SISTEMA
%----------------------------------------------------------------------------------------------
Los componentes relevantes del sistema son los que ya han sido explicados en el apartado de Aspectos Arquitecturales y Tecnológicos, propios del patrón de diseño de Modelo-Vista-Presentador. Hemos elegido este patrón de diseño porque consideramos que es un patrón que se ajusta perfectamente al problema que tenemos que resolver y a su vez ofrece una solución lo más sencilla posible.

\vspace{.2cm}
\begin{figure}[ht]
\centerline{\includegraphics[scale=0.6]{img/componentes}}\
\caption{Componentes relevantes del sistema}
\label{fig:diagCompon}
\end{figure}

%==============================================================================================
% 04.   VERIFICACIÓN Y VALIDACIÓN DEL SISTEMA
% 04.1. DESCRIPCIÓN DE LA METODOLOGÍA SEGUIDA PARA LA REALIZACIÓN DE LAS PRUEBAS DEL SISTEMA
% 04.2. PRUEBAS REALIZADAS, DEFECTOS ENCONTRADOS Y CAMBIOS Y CORRECCIONES QUE HA HABIDO
%       QUE REALIZAR. EN ESTA SECCIÓN O EN ANEXO LOS INFORMES DE TODAS LAS PRUEBAS.
%==============================================================================================

% 04
\section{Verificación y validación}
% 04. VERIFICACIÓN Y VALIDACIÓN DEL SISTEMA
%==============================================================================================
Se han utilizado las técnicas de verificación y validación conocidas por Alberto Berbel, nuestro experto de equipo en este apartado. Se han reutilizado además herramientas y procedimientos utilizados en la verificación y validación de proyectos anteriores, pues se consideraron interestantes por las características de nuestro nuevo proyecto al ser de índoles similiares.

% 04.1
\subsection{Metodología de pruebas}
% 04.1. DESCRIPCIÓN DE LA METODOLOGÍA SEGUIDA PARA LA REALIZACIÓN DE LAS PRUEBAS DEL SISTEMA
%----------------------------------------------------------------------------------------------
Puesto que para la realización de las pruebas hemos utilizado nuestros conocimientos previos adquiridos en la aplicación de Validación y Verificación en otros proyectos, para no desviarnos de lo conocido, nuestra metodología de pruebas se aproxima a la metodología de pruebas utilizada en dichos proyectos anteriores: TMAP Next (Test Management Approach).
\\[6pt]
Los fundamentos de TMAP se basan en cuatro elementos esenciales:
\begin{itemize}
	\item \textbf{Proceso dirigido por el negocio.} La economía marca el esfuerzo a realizar, y cuáles son los riesgos prioritarios.
	\item \textbf{Un proceso de pruebas estructurado.} Nos guía a la hora de responder a las cuestiones típicas de \textit{qué/cuándo, cómo, con qué y quién} (ciclo de vida de pruebas).
	\item \textbf{Un kit de herramientas.} Se ofrece información práctica para establecer la infraestructura (\textit{con qué}), las técnicas (\textit{cómo}), y la organización (\textit{quién}).
	\item \textbf{Método completo y adaptable.} Flexibilidad para adaptar la metodología a distintas situaciones de desarrollo: nuevos desarrollos, mantenimiento, desarrollo propio o basado en software comercial, ...
\end{itemize}
\paragraph{} Pasos a dar en la gestión de pruebas dirigida por el negocio:
\begin{enumerate} 
	\item Identificar los objetivos de las pruebas
	\item Determinar los riesgos
	\item Determinar si una característica/parte se debe probar de forma detallada o ligera
	\item Estimar y planificar
	\item Elegir las técnicas de prueba y ejecutarlas 
	\item Informar sobre el progreso, calidad
\end{enumerate}


% 04.2
\subsection{Pruebas. Defectos. Correcciones}
% 04.2. PRUEBAS REALIZADAS, DEFECTOS ENCONTRADOS Y CAMBIOS Y CORRECCIONES QUE HA HABIDO
%       QUE REALIZAR. EN ESTA SECCIÓN O EN ANEXO LOS INFORMES DE TODAS LAS PRUEBAS.
%----------------------------------------------------------------------------------------------

% Configuración margenes lista de requisitos
\setlist[itemize]{leftmargin=0.5in,labelindent=\parindent}


Durante el proceso de desarrollo de nuestro proyecto, los desarrolladores han ido probando que cada método que iban
implementando cumplía sus funciones y satisfacía las precondiciones y postcondiciones para las que han sido creados.
Sin embargo, hemos considerado que no era necesario documentar este tipo de pruebas puesto que las personas que nos 
han encargado el proyecto no las necesitaban, tan solo las pruebas de sistema. Además, preparar y documentar debidamente
las pruebas unitarias supone un elevadísimo coste de recursos (humanos y temporales). Por lo tanto, se puede decir que se han realizado pruebas unitarias para asegurar el correcto funcionamiento del código implementado pero no hay constancia física de ellas.

Para la realización de las pruebas de sistema, nos hemos centrado en comprobar que nuestro sistema cumple las características que nos comprometimos a desarrollar antes de comenzar este proyecto, cuando conocimos las necesidades que tenía nuestro cliente y quedaron reflejadas en la propuesta del proyecto en forma de requisitos funcionales y no funcionales del sistema.
\\[6pt]
Las pruebas de sistema que hemos realizado han sido las siguientes:
\begin{itemize}
\item \vspace{0.1in} \textbf{Primera iteración}, funcionalidades de \textbf{administración}.
	\begin{itemize}
	\item \textbf{Comprobación de la inserción de un nuevo elemento al catálogo.} Queremos comprobar que se introduce 				correctamente un nuevo elemento al catálogo y al volver a cargar el listado de elementos disponibles aparece el nuevo elemento introducido. Desde que se realizó esta prueba por primera vez, los resultados han sido los esperados.
	
	\item \textbf{Comprobación de la inserción de un nuevo elemento al catálogo sin rellenar alguno de los campos.} Queremos comprobar que el sistema no nos permite introducir un elemento incompleto al catálogo, pues todos sus campos deberían ser obligatorios. La primera vez que se realizó esta prueba, el sistema sí introducía el elemento incompleto y por lo tanto fue necesario cambiar el código implicado en esta funcionalidad. Ahora, si se intenta introducir un elemento incompleto en el catálogo, aparece un mensaje informando de que todos los campos son obligatorios y el nuevo elemento no es insertado en el catálogo.	
	
	\item \textbf{Comprobación de la modificación de un elemento existente en el catálogo.} Queremos comprobar que el sistema permite modificar las características de un elemento existente en el catálogo. Queremos comprobar que el sistema permite modificar las características de un elemento existente en el catálogo. Para ello se ejecutan pruebas modificando sus diferentes caracterísiticas (Arquitectura, Frecuencia, Flash, RAM y Precio) de manera combinada: de dos en dos, un sólo campo, todos los campos, etc. Desde la primera vez que se ejecutó esta prueba, los resultados han sido los esperados.
	
	\item \textbf{Comprobación de la eliminación de un elemento existente en el catálogo.} Queremos comprobar que el sistema permite eliminar un elemento existente en el catálogo y que por lo tanto al volver a cargar el listado del catálogo, el elemento eliminado ya no aparece en el listado. Desde la primera vez que se ejecutó esta prueba, los resultados han sido los esperados.

	\item \textbf{Comprobación del listado de los elementos del catálogo.} Queremos comprobar que el sistema muestra todos los elementos existentes en el catálogo, desde el primero hasta el último. Desde la primera vez que se ejecutó esta prueba, los resultados han sido los esperados.
	\end{itemize}
	
\item \vspace{0.1in} \textbf{Primera iteración}, funcionalidades propias del \textbf{usuario}.
	\begin{itemize}
	\item \textbf{Comprobación del listado de los elementos del catálogo.} Queremos comprobar que un usuario tiene acceso al listado de todos los elementos existentes en el catálogo, desde el primero hasta el último. Desde la primera vez que se ejecutó esta prueba, los resultados han sido los esperados.
	
	\item \textbf{Comprobación de los datos que son solicitados al cliente para generar la factura con los elementos del carro de compra.} Queremos comprobar que los datos que son solicitados al cliente son los mismos que los que acordamos con el cliente en los requisitos funcionales del sistema. Desde la primera vez que se ejecutó esta prueba, los resultados han sido los esperados.

	\item \textbf{Comprobación de la generación de la factura si el cliente no ha introducido cualquiera de los campos obligatorios.} Queremos comprobar que todos los datos requeridos al cliente deben ser introducidos para la generación de la factura. Que no permita generar la factura si hay algún campo vacío. La primera vez que se ejecutó esta prueba, el sistema sí generaba la factura aunque quedara algún campo sin completar. Por lo tanto, fue necesario modificar el código implicado en está funcionalidad, y ahora el resultado que se produce es que el sistema muestra un mensaje indicando que todos los campos son obligatorios.
	\end{itemize}		
	
\item \vspace{0.1in} \textbf{Segunda iteración}, funcionalidades de \textbf{administración}.
	\begin{itemize}
	\item \textbf{Comprobación de la inserción de un nuevo elemento al catálogo.} Queremos comprobar que se introduce 				correctamente un nuevo elemento al catálogo y al volver a cargar el listado de elementos disponibles aparece el nuevo elemento introducido. Igual que en la primera iteración, los resultados han sido los esperados.

	\item \textbf{Comprobación de la inserción de un nuevo elemento al catálogo sin rellenar alguno de los campos.} Queremos comprobar que el sistema no nos permite introducir un elemento incompleto al catálogo, pues todos sus campos deberían ser obligatorios. Como ya se corrigió durante la primera iteración, los resultados obtenidos son los esperados. 	

	\item \textbf{Comprobación de la modificación de un elemento existente en el catálogo.} Queremos comprobar que el sistema permite modificar las características de un elemento existente en el catálogo. Para ello se ejecutan pruebas modificando sus diferentes caracterísiticas (Arquitectura, Frecuencia, Flash, RAM y Precio) de manera combinada: de dos en dos, un sólo campo, todos los campos, etc. Igual que en la primera iteración, los resultados han sido los esperados.
	
	\item \textbf{Comprobación de la eliminación de un elemento existente en el catálogo.} Queremos comprobar que el sistema permite eliminar un elemento existente en el catálogo y que por lo tanto al volver a cargar el listado del catálogo, el elemento eliminado ya no aparece en el listado. Igual que en la primera iteración, los resultados han sido los esperados.
	
	\item \textbf{Comprobación del listado de los elementos del catálogo.} Queremos comprobar que el sistema muestra todos los elementos existentes en el catálogo, desde el primero hasta el último. Igual que en la primera iteración, los resultados han sido los esperados.
	
	\item \vspace{0.5in} \textbf{Comprobación de la búsqueda exitosa de elementos del catálogo por cualquiera de los campos de búsqueda.} Queremos comprobar que el sistema es capaz de realizar búsquedas en el catálogo con cada criterio de búsqueda (Arquitectura, Frecuencia, Flash, RAM). Es decir, que devuelve como resultado el listado de todos los elementos del catálogo que cumplen el criterio de búsqueda. Los resultados obtenidos en esta prueba son los esperados.
		
	\item \textbf{Comprobación de la búsqueda fallida de elementos del catálogo por cualquiera de los campos de búsqueda.} Queremos comprobar que el sistema es capaz de gestionar los casos en los que las búsquedas con cada criterio de búsquda (Arquitectura, Frecuencai, Flash, RAM) no sean satisfactorias e informe al usuario de que no hay elementos en el catálogo que coincidan con su criterio de búsqueda. Los resultados obtenidos en esta prueba son los esperados.
	\end{itemize}

\item \vspace{0.1in} \textbf{Segunda iteración}, funcionalidades propias del \textbf{usuario}.
	\begin{itemize}
	\item \textbf{Comprobación del listado de los elementos del catálogo.} Queremos comprobar que un usuario tiene acceso al listado de todos los elementos existentes en el catálogo, desde el primero hasta el último. Igual que en la primera iteración, los resultados han sido los esperados.
	
	\item \textbf{Comprobación de la búsqueda exitosa de elementos del catálogo por cualquiera de los campos de búsqueda.} Queremos comprobar que el sistema es capaz de realizar búsquedas en el catálogo con cada criterio de búsqueda (Arquitectura, Frecuencia, Flash, RAM). Es decir, que devuelve como resultado el listado de todos los elementos del catálogo que cumplen el criterio de búsqueda. Los resultados obtenidos en esta prueba son los esperados.
	
	\item \textbf{Comprobación de la búsqueda fallida de elementos del catálogo por cualquiera de los campos de búsqueda.} Queremos comprobar que el sistema es capaz de gestionar los casos en los que las búsquedas con cada criterio de búsquda (Arquitectura, Frecuencai, Flash, RAM) no sean satisfactorias e informe al usuario de que no hay elementos en el catálogo que coincidan con su criterio de búsqueda. Los resultados obtenidos en esta prueba son los esperados.

	\item \textbf{Comprobación de los datos que son solicitados al cliente para generar la factura con los elementos del carro de compra.} Queremos comprobar que los datos que son solicitados al cliente son los mismos que los que acordamos con el cliente en los requisitos funcionales del sistema. Desde la primera vez que se ejecutó esta prueba, los resultados han sido los esperados.
	
	\item \textbf{Comprobación de la generación de la factura si el cliente no ha introducido cualquiera de los campos obligatorios.} Queremos comprobar que todos los datos requeridos al cliente deben ser introducidos para la generación de la factura. Que no permita generar la factura si hay algún campo vacío. Como ya se corrigió durante la primera iteración, los resultados obtenidos son los esperados.
	\end{itemize}			
\end{itemize}
\paragraph{}Se puede acceder en el anexo \ref{sec:informe_pruebas} al informe completo y detallado de las pruebas.

%==============================================================================================
% 05.   MANUALES
% 05.1. MANUAL DE USUARIO
% 05.1. MANUAL DE INSTALACIÓN
%==============================================================================================

% 05
\section{Manuales}
% 05. MANUALES
%==============================================================================================

\paragraph{}En este apartado se presentan los manuales de usuario necesarios para que tanto el usuario cliente de la aplicación como el administrador de la misma aprendan a utilizarla y navegar por ella de la forma más rápida y sencilla posible.

\paragraph{}Además se añade un manual de instalación para que, en caso de que la empresa cliente a la que va dirigida este producto decida mantener ella el catálogo electrónico, pueda instalar y preparar el entorno necesario para ello.

% 05.1
\subsection{Manual de usuario}
% 05.1. MANUAL DE USUARIO
%----------------------------------------------------------------------------------------------
El manual de usuario, incluye dos versiones conjuntas en un mismo documento:
\begin{itemize}
	\item La del usuario cliente que utilzará la aplicación para navegar por la misma y realizar sus compras de microcontroladores
	\item La del usuario administrador de la empresa cliente que tendrá acceso interno al catálogo electronico para añadir, modificar y eliminar microcontroladores del mismo.
\end{itemize}

Se puede acceder a él a través del \manualUsuario{guia_usuario}{siguiente enlace}.


% 05.2
\subsection{Manual de instalación}
% 05.2. MANUAL DE INSTALACIÓN
%----------------------------------------------------------------------------------------------
\paragraph{}Dado que nuestra empresa, \textit{\textbf{BitParty}}, ofrece el servicio de encargarse de lanzar y poner en funcionamiento el catálogo electrónico y del mantenimiento del servidor y la base de datos que permitirán a la aplicación estar totalmente operativa \textit{online} y disponible 24 horas al día, el cliente no deberá preocuparse por como instalar la aplicación, ni poner en marcha un servidor, etc...

\paragraph{}Así pues en un principio el cliente sólo deberá preocuparse de abastecernos con la información completa de su catálago de microcontroladores.

\paragraph{}Aún así, si en un futuro el cliente se ve capacitado para realizar dichas tareas de lanzamiento y  mantenimiento de la aplicación, gustosamente se le genererá y entregará un manual de instalación para facilitarle dicha tarea.

%==============================================================================================
% 06.     GESTIÓN DEL PROYECTO
% 06.1.   FASES Y ACTIVIDADES DEL PROYECTO
% 06.1.1. RIESGOS
% 06.1.2. ESTIMACIONES GLOBALES DE TAMAÑOS Y ESFUERZOS INICIALES
% 06.1.3. CRONOGRAMAS GLOBAL INICIAL Y FINAL
% 06.1.4. TAREAS Y ESTIMACIONES DE ESFUERZOS POR ITERACIÓN 
% 06.1.5. FICHEROS DE ESFUERZOS INDIVIDUALES 
% 06.1.6. ESFUERZOS REALES DE LAS TAREAS POR ITERACIÓN 
% 06.1.7. ESFUERZOS REALES DE LAS PERSONAS Y ROLES 
% 06.2.   PROCESOS DE SEGUIMIENTO Y CONTROL
% 06.2.1. CALENDARIO DE LAS DISTINTAS REUNIONES CELEBRADAS
% 06.2.2. ACTAS DE LAS DISTINTAS REUNIONES CELEBRADAS
% 06.3.   COSTE REAL DEL PROYECTO
%==============================================================================================

% 06
\section{Gestión del proyecto}
% 06. GESTIÓN DEL PROYECTO
%----------------------------------------------------------------------------------------
En esta parte se ha planificado, organizado, monitorizado y controlado el proyecto. 
El objetivo de todo esto es disminuir los costes, satisfacer las fechas de entrega, conseguir un buen producto y prever adecuadamente los posibles problemas que puedan surgir a lo largo de todo el proceso.

% 06.1
\subsection{Fases y actividades}
% 06.1. FASES Y ACTIVIDADES DEL PROYECTO
%----------------------------------------------------------------------------------------
En esta sección se explican las diferentes fases en las que se ha dividido y por las que ha pasado el proyecto, cómo se han planificado, qué tareas se han realizado en cada una de ellas y que esfuerzos globales han supuesto llevarlas a cabo.

% 06.1.1
\subsubsection{Riesgos}\label{subsec:riesgos}
% 06.1.1. RIESGOS
%----------------------------------------------------------------------------------------
Se han identificado, definido y clasificado los posibles riesgos que el equipo se podía encontrar durante el desarrollo del proyecta; estableciendo además las consiguientes estrategias de respuesta y mitigación para cada uno de ellos.
En el \riesgos{registro_de_riesgos}{siguiente enlace} se puede acceder al catálogo donde se encuentran.


% 06.1.2
\subsubsection{Estimaciones iniciales}
% 06.1.2. ESTIMACIONES GLOBALES DE TAMAÑOS Y ESFUERZOS INICIALES
%----------------------------------------------------------------------------------------

\paragraph{} Se muestra en la~\cref{fig:6121} los tiempos que se han asignado a las diferentes tareas. En esta figura se muestran las horas concretas que se estimaron al principio del proyecto. Más adelante se mostrara cuantas horas reales se invirtieron en cada apartado.

\begin{figure}[h!]
\centering
\includegraphics[width=0.95\textwidth]{img/6121}
\caption{Estimación esfuerzos totales}
 \label{fig:6121}
\end{figure}

\paragraph{} Se muestra en la~\cref{fig:6122} las horas asignadas a cada una de las tareas. Se puede ver que las partes más costosas son sobre todo el lanzamiento del proyecto,gestión y pruebas.Implementación se llevaría otra de las grandes partes del proyecto, pero al estar dividida en subtareas parece que ocupe un tiempo menor del que en realidad lleva.

\begin{figure}[h!]
\centering
\includegraphics[width=0.95\textwidth]{img/6122}
\caption{Gráfica esfuerzos totales}
 \label{fig:6122}
\end{figure}

% 06.1.3
\subsubsection{Cronogramas}
% 06.1.3. CRONOGRAMAS GLOBAL INICIAL Y FINAL
%----------------------------------------------------------------------------------------

\paragraph{} Se pueden visualizar, tanto el cronograma que se planificó inicialmente y como el cronograma que resulto finalmente, en el anexo \ref{sec:cronogramas} de este documento. 
\paragraph{}En general las tareas se retrasaron un poco frente a la planificación prevista, tanto su comienzo como su finalización. Pero siempre se ha ido desarrollando el proyecto con los tiempos previstos y cumpliendo los objetivos adecuadamente. Alguna de las tareas se pudo alargar algo más en el tiempo que lo mostrado en el cronograma, pero debido siempre a pequeñas modificaciones de última hora o detalles olvidados que no eran relevantes. 



% 06.1.4
\subsubsection{Tareas y estimación de esfuerzos}
% 06.1.4. TAREAS Y ESTIMACIONES DE ESFUERZOS POR ITERACIÓN 
%----------------------------------------------------------------------------------------


\paragraph{} Se muestran los esfuerzos previstos para la primera iteración del proyecto en la figura \ref{fig:6141}.

\begin{figure}[h!]
\centering
\includegraphics[width=0.95\textwidth]{img/6141}
\caption{Esfuerzos primera iteración}
 \label{fig:6141}
\end{figure}

\paragraph{} Se puede apreciar como la parte más importante de la primera iteración fue el lanzamiento, en el cual se planifico todo el proyecto para concretar todo e intentar evitar posibles problemas. Entre las demás partes se planifico que pruebas, gestión y implementación serías las siguientes tareas más costosas. Esto se puede ver en la~\cref{fig:6142}.

\begin{figure}[h!]
\centering
\includegraphics[width=0.95\textwidth]{img/6142}
\caption{Gráfica esfuerzos primera iteración}
 \label{fig:6142}
\end{figure}

\paragraph{} Se muestran los esfuerzos previstos para la segunda iteración del proyecto en la~\cref{fig:6143}. Podemos observar que la mayor parte de horas se la llevan la mejora de la implementación de la interfaz web (con la intención de que la misma tenga un aspecto llamativo y atractivo para el cliente) y las pruebas (con la intención de asegurar completamente que el producto que se le entrega al cliente funciona correctamente en su totalidad). Podemos ver dicho reparto de trabajo más claramente en la~\cref{fig:6144}. 

\begin{figure}[h!]
\centering
\includegraphics[width=0.95\textwidth]{img/6143}
\caption{Esfuerzos segunda iteración}
 \label{fig:6143}
\end{figure}

\begin{figure}[h!]
\centering
\includegraphics[width=0.95\textwidth]{img/6144}
\caption{Gráfica esfuerzos segunda iteración}
 \label{fig:6144}
\end{figure}

% 06.1.5
\subsubsection{Ficheros de esfuerzos individuales}
% 06.1.5. FICHEROS DE ESFUERZOS INDIVIDUALES 
%----------------------------------------------------------------------------------------
Las hojas de esfuerzos individuales para cada uno de los miembros del equipo se encuentran disponibles en el Anexo \ref{sec:hojas_esfuerzos} de este documento.

% 06.1.6
\subsubsection{Esfuerzos reales de las tareas}\label{sec:esfuerzos_reales}
% 06.1.6. ESFUERZOS REALES DE LAS TAREAS POR ITERACIÓN 
%----------------------------------------------------------------------------------------

\paragraph{} Para mostrar los esfuerzos reales de cada iteración se han realizado varias gráficas.

\paragraph{} Como se aprecia en la ~\cref{fig:6161} el mes de más trabajo fue Marzo. Además se muestran las horas invertidas en el proyecto y concretamente en cada una de las tareas. En total se realizaron 274 horas frente a las 339 previstas. Pero estas 339 son con la corrección aplicada al desconocimiento de las mismas al inicio del proyecto. Sin esa corrección se estimaron 212 por lo que se han mantenido las horas entre ambas.

\begin{figure}[h!]
\centering
\includegraphics[width=0.95\textwidth]{img/6161}
\caption{Esfuerzos reales primera iteración}
 \label{fig:6161}
\end{figure}

\paragraph{} En la figura ~\cref{fig:6162} se muestra lo anterior más gráficamente. La ultima columna sería el global de la primera iteración y las demás las subtareas.

\begin{figure}[h!]
\centering
\includegraphics[width=0.85\textwidth]{img/6162}
\caption{Esfuerzos reales primera iteración}
 \label{fig:6162}
\end{figure}


\paragraph{} Para la segunda iteración se muestran la misma gráficas utilizadas para el análisis de la primera iteración pero con las horas invertidas en la segunda iteración (~\cref{fig:6163} y ~\cref{fig:6164}).

\begin{figure}[h!]
\centering
\includegraphics[width=0.95\textwidth]{img/6163}
\caption{Esfuerzos reales segunda iteración}
 \label{fig:6163}
\end{figure}

\begin{figure}[h!]
\centering
\includegraphics[width=0.95\textwidth]{img/6164}
\caption{Esfuerzos reales segunda iteración}
 \label{fig:6164}
\end{figure}

% 06.1.7
\subsubsection{Esfuerzos reales de las personas}
% 06.1.7. ESFUERZOS REALES DE LAS PERSONAS Y ROLES 
%----------------------------------------------------------------------------------------

\paragraph{} Se muestran los datos y las gráficas con las horas realizadas por las diferentes personas del equipo en las figuras ~\cref{fig:6171} y ~\cref{fig:6172}. 

\paragraph{} Los roles de los miembros del grupo son los siguientes:
\begin{itemize}
\item Daniel $\Rightarrow$ Director de proyecto
\item Alberto $\Rightarrow$ Verificación y validación
\item Javier $\Rightarrow$ Gestor de configuraciones
\item Héctor $\Rightarrow$ Gestor de calidad
\item Simón $\Rightarrow$ Gestor de desarrollo
\item Alejandro $\Rightarrow$ Gestor de planificación
\end{itemize}

\paragraph{} En la gráfica ~\cref{fig:6171} se ven las diferentes horas invertidas por los integrantes del grupo. Algún aspecto a destacar es que Daniel, como director del proyecto, invirtió el que más horas en el mismo. Además el mes de Marzo, como ya se ha recalcado anteriormente, fue en el que más horas se invirtieron, es decir, ese mes fue el grueso del proyecto. También podemos observar como Javier, al ser el principal encargado de la implementación de la aplicación, tenía una carga de trabajo inicial menor pero después mientras el resto del grupo bajaba el nivel de trabajo él lo subía.

\begin{figure}[h!]
\centering
\includegraphics[width=0.95\textwidth]{img/6171}
\caption{Esfuerzos reales por persona primera iteración}
 \label{fig:6171}
\end{figure} 

\paragraph{} Se presentan a continuación los datos para la segunda iteración. De nuevo el director, Daniel, invirtió el que más horas en el proyecto y os demás integrantes del grupo más o menos hicieron las mismas horas de trabajo. Datos que representan una buena organización y reparto de trabajo.

\begin{figure}[h!]
\centering
\includegraphics[width=0.85\textwidth]{img/6172}
\caption{Esfuerzos reales por persona primera iteración}
 \label{fig:6172}
\end{figure} 

% 06.2
\subsection{Procesos de seguimiento y control}
% 06.2. PROCESOS DE SEGUIMIENTO Y CONTROL
%----------------------------------------------------------------------------------------
Para el seguimiento y control del estado en el que se encontraba el desarrollo de nuestro proyecto se han ido realizando reuniones de todo los miembros del grupo cada semana o cada dos semanas. En ellas se analizaba el trabajo realizado hasta ese momento, se planificaban y repartían nuevas tareas y se resolvían posibles dudas o problemas que pudieran surgir.

% 06.2.1
\subsubsection{Calendario de reuniones}
% 06.2.1. CALENDARIO DE LAS DISTINTAS REUNIONES CELEBRADAS
%----------------------------------------------------------------------------------------
Se muestra a continuación el calendario de las reuniones que el equipo ha realizado durante el desarrollo el proyecto. Cada reunión tiene identificador propio, nombre, fecha y hora:
\begin{itemize}
\item R-01 Lanzamiento de equipo: 	 20/02/2014 - 10:00
\item R-02 Lanzamiento de proyecto: 20/02/2014 - 12:00
\item R-03 Lanzamiento de proyecto: 26/02/2014 - 20:00
\item R-04 Seguimiento de proyecto y 1$^a$ iteración: 06/03/2014 - 10:00
\item R-05 Práctica 3, Comienzo primera iteración: 06/03/2014 - 12:00
\item R-06 Reunión de seguimiento semanal: 10/03/2014 - 20:00
\item R-07 Reunión interna de seguimiento: 12/03/2014 - 20:00 
\item R-08 Reunión de seguimiento: 27/03/2014 10:00
\item R-09 Reunión para Aseguramiento de la Calidad del proyecto: 27/03/2014 -12:00
\item R-10 Reunión de seguimiento: 09/04/2014 -20:00
\item R-11 Reunión final de 1$^a$ iteración y comienzo de la 2$^a$: 09/04/2014 - 20:00
\end{itemize}


% 06.2.2
\subsubsection{Actas de reuniones}
% 06.2.2. ACTAS DE LAS DISTINTAS REUNIONES CELEBRADAS
%----------------------------------------------------------------------------------------
Se muestran a continuación los enlaces directos a las actas de cada una de las reuniones
\begin{itemize}
%\item R-01 Lanzamiento de equipo -- Anexo \ref{subsec:r1}
\item \reuniones{r01_lanzamiento_del_equipo/acta/r01_e20_acta}{R-01 Lanzamiento de equipo}
%\item R-02 Lanzamiento de proyecto -- Anexo \ref{subsec:r2}
\item \reuniones{r02_lanzamiento_del_proyecto/acta/r02_e20_acta}{R-02 Lanzamiento del proyecto}
%\item R-03 Lanzamiento de proyecto -- Anexo \ref{subsec:r3}
\item \reuniones{r03_lanzamiento_del_proyecto_cont/acta/r03_e20_acta}{R-03 Lanzamiento del proyecto II}
%\item R-04 Seguimiento de proyecto y 1$^a$ iteración -- Anexo \ref{subsec:r4}
\item \reuniones{r04_seguimiento_propuesta_proyecto/acta/r04_e20_acta}{R-04 Seguimiento de la propuesta de proyecto}
%\item R-05 Práctica 3, Comienzo primera iteración -- Anexo \ref{subsec:r5}
\item \reuniones{r05_comienzo_1a_iteracion/acta/r05_e7_acta}{R-05 Comienzo 1$^a$ iteración}
%\item R-06 Reunión de seguimiento semanal -- Anexo \ref{subsec:r6}
\item \reuniones{r06_reunion_seguimiento/acta/r06_e20_acta}{R-06 Reunión de seguimiento del proyecto}
%\item R-07 Reunión interna de seguimiento -- Anexo \ref{subsec:r7}
\item \reuniones{r07_reunion_seguimiento/acta/r07_e7_acta}{R-07 Reunión de seguimiento del proyecto}
%\item R-08 Reunión de seguimiento -- Anexo \ref{subsec:r8}
\item \reuniones{r08_reunion_seguimiento/acta/r08_e20_acta}{R-08 Reunión de seguimiento del proyecto}
%\item R-09 Reunión para Aseguramiento de la Calidad del proyecto -- Anexo \ref{subsec:r9}
\item \reuniones{r09_aseguramiento_calidad_proyecto/acta/r09_e20_acta}{R-09 Aseguramiento de calidad del proyecto}
%\item R-10 Reunión de seguimiento -- Anexo \ref{subsec:r10}
\item \reuniones{r10_reunion_seguimiento/acta/r10_e20_acta}{R-10 Reunión de seguimiento del proyecto}
%\item R-11 Reunión final de 1$^a$ iteración y comienzo de la 2$^a$ -- Anexo \ref{subsec:r11}
\item \reuniones{r11_final_1a_iteracion/acta/r11_e20_acta}{R-11 Reunión final de 1$^a$ iteración y comienzo de la 2$^a$}
\end{itemize}

% 06.3
\subsection{Coste}\label{subsec:coste}
% 06.3. COSTE REAL DEL PROYECTO
%----------------------------------------------------------------------------------------

\paragraph{} Para analizar el coste real del proyecto, se han sumado las horas realizadas en  cada una de las iteraciones: 274 horas + 159 horas = 433 horas en total. No se han comprado ni servidores ni han sido necesarios gastos extras.  

\paragraph{} Por lo que el presupuesto final son 8.010,50\euro. que sumando el IVA son : 9.692,2\euro.

\paragraph{} Como se ha realizado el proyecto en menos horas que las estimadas y se aplicó un buen margen de beneficio, se ha conseguido un gran beneficio del proyecto para la empresa que asciende a 8.388,62\euro.

\paragraph{}En los siguientes enlaces se puede acceder a:
\begin{itemize}
	\item \presupuestos{presupuesto_iteracion1}{El cálculo del presupuesto para la primera iteración.}
	\item \presupuestos{presupuesto_iteracion2}{El cálculo del presupuesto para la segunda iteración.}
	\item \presupuesto{oferta_economica}{La oferta final del presupuesto que se realizó.}
\end{itemize}

%==============================================================================================
% 07.   GESTIÓN DE CONFIGURACIONES DEL PROYECTO
% 07.1. POLÍTICAS DE NOMBRADO
% 07.2. CONTROL DE VERSIONES
% 07.3. COPIAS DE SEGURIDAD
% 07.4. ELEMENTOS DE CONFIGURACIÓN Y LINEA BASE
%==============================================================================================

% 07
\section{Gestión de configuraciones}
% 07. GESTIÓN DE CONFIGURACIONES DEL PROYECTO
%==============================================================================================


% 07.1
\subsection{Políticas de nombrado}
% 07.1. POLÍTICAS DE NOMBRADO
%----------------------------------------------------------------------------------------------
El nombrado de ficheros debe realizarse en minúsculas, sin espacios, pudiendo utilizar el carácter '\_' en su lugar.

El nombre de cada fichero debe finalizar con una extensión separada del nombre del fichero por un punto, siendo la extensión en minúsculas y apropiada al tipo de contenido del fichero (php, png, html, txt, pdf...).

% 07.2
\subsection{Control de versiones}
% 07.2. CONTROL DE VERSIONES
%----------------------------------------------------------------------------------------------
El control de versiones utilizado será subversion, utilizándose como servicio de almacenamiento el de Google: Google Code, dada su elevada disponibilidad, siendo de esta manera accesible el proyecto para todos los miembros del equipo.


% 07.3
\subsection{Copias de seguridad}
% 07.3. COPIAS DE SEGURIDAD
%----------------------------------------------------------------------------------------------
Durante el desarrollo del proyecto se ha dispuesto en todo momento de una copia del sistema, operativa, a fin de poder llevar a cabo pruebas con el sistema online. Esta copia ha jugado un doble papel: el de servir como copia de seguridad en caso de no poder acceder a Google Code y el de servir como copia viva del entorno.

% 07.4
\subsection{Elementos de configuración y línea base}
% 07.4. ELEMENTOS DE CONFIGURACIÓN Y LINEA BASE
%----------------------------------------------------------------------------------------------
Durante el desarrollo del proyecto se generarán diversos documentos que conformarán la configuración del proyecto.

En primer lugar se establecerá un marco de trabajo, anteriormente descrito, que establecerá las políticas de nombrado, formatos utilizados, plataforma de control de versiones, y otros elementos que harán posible el trabajo en equipo y el intercambio de información.

El primer documento generado de cara al cliente será la propuesta de proyecto, junto con el presupuesto del mismo. Este documento recoge un plan de proyecto, la definición de requisitos, un primer análisis y una descripción técnica de la propuesta de solución. También se integra en esa primera propuesta la composición del equipo que llevará a cabo el proyecto.

Otros documentos que conforman la configuración del proyecto son los documentos de análisis y pruebas del sistema, el prototipo, los documentos de diseño e imagen del proyecto, el código fuente del programa, los documentos de calidad, los manuales, y otros documentos disponibles en el control de versiones del proyecto.

Se establece como linea base del programa, el resultado de la primera iteración del desarrollo, a partir de la cual se inicia la segunda iteración a cuya finalización se deberían haber completado los requisitos planteados por el cliente.

%==============================================================================================
% 08.   ASEGURAMIENTO DE LA CALIDAD DEL PROYECTO
% 08.1. ESTÁNDARES UTILIZADOS
% 08.2. PLANIFICACIÓN DE LAS AUDITORÍAS
% 08.3. AUDITORÍAS
% 08.4. AUDITORÍA EXTERNA
%==============================================================================================

% 08
\section{Aseguramiento de la calidad}
%% Véase el~\cref{anex:Auditorias}.
% 08. ASEGURAMIENTO DE LA CALIDAD DEL PROYECTO
%==============================================================================================
Para asegurar la calidad de nuestro proyecto nuestros objetivos han sido mejorar el software monitorizándolo apropiadamente y también el proceso de desarrollo que lo produce. 
De esta manera se busca asegurar la completa concordancia con los estándares y procedimientos establecidos para nuestro software y nuestro proceso de desarrollo.
Con ello conseguimos que cualquier elemento que no sea adecuado en el producto, el proceso, o los estándares sea puesto en conocimiento de los responsables para que pueda ser resuelto.


% 08.1
\subsection{Estándares utilizados}
% 08.1. ESTÁNDARES UTILIZADOS
%----------------------------------------------------------------------------------------------
Como estándar utilizado a lo largo de todo el proceso se ha seguido una \textbf{guía de codificación} definida al principio del proyecto y de la cual no nos hemos apartado.
Se compone de las siguientes pautas:

\begin{itemize}
	\item Longitud máxima de línea recomendada, 80 carácteres.
	\item Métodos/funciones de un máximo recomendado de 30 líneas.
	\item Nombres de clase en mayúscula.
	\item Nombres de métodos/funciones en minúscula.
	\item Tabulado con dos espacios.
	\item Codificación UTF-8 en todos los ficheros de texto plano.
	\item Es importante respetar el indentado en el código fuente php/html "crudo" pero no se tratará de que el código esté correctamente indentado tras pasar por el intérprete php.
\end{itemize}










% 08.2
\subsection{Planificación de las auditorías}
% 08.2. PLANIFICACIÓN DE LAS AUDITORÍAS
%----------------------------------------------------------------------------------------------
\begin{itemize}
	\item La \textbf{auditoría interna} se planificó para el día 7 de Mayo de 2014 y se estimó una duración de 1 hora. Finalmente se llevó a cabo ese día pero la duración fue de 50 minutos.
	\item La \textbf{auditoría externa} se planificó para el día 9 de Mayo de 2014 y se estimó una duración de 1 hora. Finalmente se llevó a cabo ese día pero la duración fue de 80 minutos.
\end{itemize}

% 08.3
\subsection{Auditorías}
% 08.3. AUDITORÍAS
%----------------------------------------------------------------------------------------------
\newcommand{\audiInterna}[2]{\href{run:../../auditorias/interna/#1.xls}{#2}}
\paragraph{}Para el seguimiento de la calidad del equipo de desarrollo, se ha comprobado el seguimiento de las bases establecidas al inicio del proyecto con una herramienta clave como son las auditorías.
\paragraph{}Éste es el método principal para validar la calidad de un proceso o de un producto ya que realiza un examen de parte o todo de ese sistema y de su documentación para encontrar problemas potenciales.

\paragraph{}En el \audiInterna{auditoria_interna}{siguiente enlace} se puede encontrar la auditoría interna realizada a nuestro proyecto.

% 08.4
\subsection{Auditoría externa}
% 08.4. AUDITORÍA EXTERNA
%----------------------------------------------------------------------------------------------
\newcommand{\audiExterna}[2]{\href{run:../../auditorias/externa/#1.xls}{#2}}
\paragraph{}En el \audiExterna{auditoria_externa}{siguiente enlace} se puede encontrar la auditoría externa realizada a nuestro proyecto.

%==============================================================================================
% 09.   POSTMORTEM DEL PROYECTO
% 09.1. LECCIONES APRENDIDAS
% 09.2. PROBLEMAS ENCONTRADOS
% 09.3. CATÁLOGO DE RIESGOS
% 09.4. ANÁLISIS DIFERENCIAS ESFUERZOS Y TAMAÑOS REALES DEL PROYECTO VS LOS ESTIMADOS
% 09.5. PLAN REAL VS PLANIFICACIÓN INCIAL
% 09.6. ANÁLISIS DIFERENCIAS COSTE REAL DEL PROYECTO VS PRESUPUESTO
%==============================================================================================

% 09
\section{Postmortem}
% 09. POSTMORTEM DEL PROYECTO
%==============================================================================================
Análisis postmortem del desarrollo del proyecto.

% 09.1
\subsection{Lecciones aprendidas}
% 09.1. LECCIONES APRENDIDAS
%----------------------------------------------------------------------------------------------
Se enumeran aquí alguna de las lecciones aprendidas por los miembros del equipo durante el desarrollo del proyecto:
\begin{itemize}
	\item Se ha aprendido la importancia del reparto de roles. Si bien la dimensión del proyecto no era quizás lo suficientemente grande para sacarle el máximo beneficio a esto, sí que sin este tipo de organización hubiera sido mucho más díficil llevar todo a cabo. Teniendo cada miembro un papel bien definido dentro del proyecto es mucho más fácil el realizar todas las tareas y juntar el trabajo periódicamente.
	\item A pesar de tener varios canales de comunicación entre los miembros del equipo (\textit{Whatsapp}, correo electrónico (\textit{Google Groups})...), es muy importante las reuniones periódicas para poner en común el trabajo. Es decir, en persona y cara a cara es mucho más efectivo trabajar, es muy importante para el desarrollo del proyecto.
	\item Se ha aprendido la importancia de buscar siempre varias alternativas a los problemas que surgen, poner ideas en común y llegar a un acuerdo como grupo. Por ejemplo, la decisión final de realizar el proyecto implementando la el catálogo con un \textit{Framework} como \textit{CodeIgniter} fue una decisión consensuada y finalmente vital para realizar a tiempo el proyecto. Así como la decisión de utilizar \textit{LaTex} para la documentación, decisión consensuada también y que ha agilizado mucho todo el desarrollo del proyecto.
	\item Se ha aprendido la importancia de que los miembros del equipo no sólo se ciñan a su papel, sino que también estén al día como mínimo de otros aspectos del proyecto. Pues si algún día faltaba un miembro o no estaba disponible durante algún tiempo, alguien tenía que suplirle temporalmente. De igual manera, es importante que el director esté al tanto de todo y se interese por el trabajo del resto de miembros del equipo.
	\item Se ha aprendido la importancia de planificar un proyecto antes de comenzarlo. Sí aún planificándolo ha habido momentos de cierta descordinación, sin la planificación pudiera haber sido un desastre que se ha evitado.
	\item Se ha aprendido la importancia de utilizar una herramienta de control de versiones como \textit{SubVersion} para desarrollo del proyecto, pudiendo deshacer cambios y errores, teniendo al instante los cambios realizados por otro compañero, etc. Además de aprender a utilizar este tipo de herramientas para futuros proyectos.
\end{itemize}
\paragraph{}En resumen, estás son algunas de las lecciones más importantes aprendidas, pero ha sido de gran importancia el aprender en general como trabajar en grupo con roles definidos y planificando cada aspecto de un proyecto.

% 09.2
\subsection{Problemas encontrados}
% 09.2. PROBLEMAS ENCONTRADOS
%----------------------------------------------------------------------------------------------
Se enumeran aquí algunos de los problemas que el equipo se ha ido encontrado durante el desarrollo del proyecto:

\begin{itemize}
	\item Aprendizaje para manejar el control de versiones de \textit{SubVersion}. Hubo problemas al inicio cuando había conflictos de modificación de ficheros.
	\item Aprendizaje de PHP por el resto de integrantes del equipo, pues sólo teníamos un experto en dicho lenguaje.
	\item Encontrar días y horas para realizar las reuniones y que todos los integrantes del equipo pudieran acudir.
\end{itemize}

\paragraph{} En general se podría decir que el proyecto se ha ido desarrollando sin muchos problemas. La mayoría de poca importancia, lo cual es un punto a favor de la organización del equipo.


% 09.3
\subsection{Catálogo de riesgos}
% 09.3. CATÁLOGO DE RIESGOS
%----------------------------------------------------------------------------------------------
\paragraph{}Como ya se ha indicado en el punto \ref{subsec:riesgos} de este documento, se han identificado, definido y clasificado los posibles riesgos que el equipo se podía encontrar durante el desarrollo del proyecta; estableciendo además las consiguientes estrategias de respuesta y mitigación para cada uno de ellos.
En el \riesgos{registro_de_riesgos}{siguiente enlace} se puede acceder al catálogo donde se encuentran.

\paragraph{}Analizándolos postmortem, se podria decir que sólo hemos tenido que lidiar con la falta de experiencia con la metodología y algunos de los elementos de planificación impuestos para el desarrollo del proyecto; así como con que sólo algunos integrantes del equipo poseían los ocnocimientos suficientes para desarrollar ciertas partes del proyecto, especialmente la implementación. Pero esto se ha solucionado con el añadido de horas extras por el resto de miembros del equipo para con el máximo esfuerzo adquirir nuevos conocimientos y así mitigar estos riesgos

% 09.4
\subsection{Diferencias entre los esfuerzos y tamaños reales y los estimados}
% 09.4. ANÁLISIS DIFERENCIAS ESFUERZOS Y TAMAÑOS REALES DEL PROYECTO VS LOS ESTIMADOS
%----------------------------------------------------------------------------------------------
\paragraph{}Como se ha visto en el apartado \ref{sec:esfuerzos_reales} de este documento, en el desarrollo del proyecto en general el equipo ha realizado menos horas de esfuerzo de las estimadas en la planificación.

\paragraph{}Esto es debido principalmente a que el equipo ya contaba con experiencia anterior en otros proyectos de esta índole en el apartado de implementación y pruebas principalmente, con lo que al finalel trabajo se ha realizado quizás en estos apartados con mayor soltura de la esperada.

\paragraph{}Aún así, si nos fijamos en esas gráficas (~\cref{fig:6162} y ~\cref{fig:6164}) se puede comprobar que el total de horas reales empleadas se encuentra en un termino medio entre las estimadas y las estimadas con el factor de corrección aplicado. Es decir, que simplemente aplicando un factor de correción menor (igual fue demasiado alto por el miedo a quedarnos cortos y el desconocimiento de nunca haber realzado antes una planificación de este tipo) las horas estimadas y reales hubieran estado a un nivel muy similar. Por lo tanto, damos por satisfactorio la diferencia final entre dichas horas de esfuerzo y nos servirá para próximos proyectos.

% 09.5
\subsection{Plan real vs planificación inicial}
% 09.5. PLAN REAL VS PLANIFICACIÓN INCIAL
%----------------------------------------------------------------------------------------------
\paragraph{}Siguiendo el análisis del punto anterior podemos decir de nuevo que en general la planificación inicial se asemeja bastante al plan real del proyecto:

\begin{itemize}
	\item Se han cumplido todos los requisitos que se impusieron al principio para el catálogo, sin grandes dificultades.
	
	\item La diferencia entre horas planificadas de esfuerzo y las reales es bastante aceptable, siendo importante que no se han sobrepasado sino que se han quedado algo cortas. Es decir, señal de equipo efectivo y de que el proyecto ha ido bien.
	
	\item Se han seguido dentro de lo posible las fechas para las tareas de cada iteración, como se puede ver en los en los cronogramas del anexo \ref{sec:cronogramas}. Ha habido pequeños desajustes, pero que no han afectado de manera grave en ningún momento a desarrollo del proyecto.
	
	\item No ha habido que lidiar prácticamente con ninguno de los riesgos que se catalogaron al principio del proyecto.
	
	\item Aunque no había una estimación inicial sobre ello, quizás sí que se han realizado más reuniones de las que uno pudiera pensar para un proyecto de tan poco tiempo. Pero siempre siendo aprovechadas.
\end{itemize}

\paragraph{}En resumen, repetimos nos consideramos satisfechos con el resultado final respecto a la planificación inicial en todos los aspectos, pues no ha habido una gran diferencia.


% 09.6
\subsection{Diferencias entre el coste real y el presupuesto}
% 09.6. ANÁLISIS DIFERENCIAS COSTE REAL DEL PROYECTO VS PRESUPUESTO
%----------------------------------------------------------------------------------------------
\paragraph{}Como ya se explica en el apartado \ref{subsec:coste} de este documento, y siguiendo el análisis de que las horas reales empleadas en el proyecto han sido menos que las estimadas inicialmente, pues evidentemente el coste real del proyecto es menor al estimado en el presupuesto. Esto daría como resultado final un beneficio a nuestra empresa de 8.388,62\euro, es decir, un resultado del que no nos podríamos quejar.

\paragraph{} De todas formas, esto también nos sirve para próximos proyectos regular el factor de correción aplicado al proyecto, quizás como ya se ha dicho antes, ha sido algo más elevado de lo que debiera, de ahí la diferencia económica a nuestro favor.

%==============================================================================================
% 10. CONCLUSIONES
% 10.1. CONCLUSIONES DEL PROYECTO
% 10.2. IDEAS DE MEJORA DEL PROCESO
% 10.3. IDEAS DE MEJORA DEL DESARROLLO DEL PROYECTO DENTRO DE LA ASIGNATURA
% 10.4. VALORACIONES SUBJETIVAS PERO ARGUMENTADAS
%==============================================================================================

% 10
\section{Conclusiones}
% 10. CONCLUSIONES
%==============================================================================================
Se presentan en este apartado las conclusiones, ideas de mejora del desarrollo del proyecto y las valoraciones subjetivas del equipo, como grupo y no de forma individual, es decir, son conclusiones, ideas y valoraciones consensuadas.

% 10.1
\subsection{Conclusiones del proyecto}
% 10.1. CONCLUSIONES DEL PROYECTO
%----------------------------------------------------------------------------------------------
Como conclusiones generales del proyecto todos los miembros del grupo concidimos en que:
\begin{itemize}
	\item  Ha sido una experiencia enriquecedora en el aspecto de aprender a trabajar en grupo, tener una organización y jerarquía elaborada, cada miembro con su papel y tareas bien definidas. No es fácil ni estamos acostumbrados a trabajar con este nivel de organización. Aprender a trabajar como equipo, siempre teniendo en cuenta la opinión de todos e intentar llegar a la mejor idea de forma consensuada. Además, por suerte podemos decir en nuestro caso que todos los miembros del equipo han rendido como se esperaba y no hemos tenido que lidiar con ese tipo de problemas. Aunque no es bueno acostumbrarse a ello pues en la vida real no será tan idílico seguramente.
	
	\item Se destaca la importancia de la planificación inicial del proyecto, especialmente en la planificación de tareas, fechas para cada tarea y reparto y asignación de tareas entre los diferentes miembros del equipo. Todo ello facilita el desarrollo del proyecto.
	
	\item También sacamos como conclusión que es importante monitorizar las horas de esfuerzo de cada miembro del equipo, así siempre se sabe quien trabaja más o menos, pudiendo repartir las tareas en las diferentes iteraciones del proyecto en función a esas horas.
\end{itemize}

En resumen, las conclusinoes del proyecto son positivas mayormente. El poder elegir al inicio el tipo de aplicación que deseabamos desarrollar ha permitido que la carga de trabajo fuese razonable durante el proyecto y ello ha ayudado a centrarnos más en la parte de organización, planificación, gestión, etc... que más interesaba en esta asignatura.


% 10.2
%\subsection{Ideas de mejora del proceso}
% 10.2. IDEAS DE MEJORA DEL PROCESO
%----------------------------------------------------------------------------------------------


% 10.3
\subsection{Ideas de mejora del desarrollo del proyecto dentro de la asignatura}
% 10.3. IDEAS DE MEJORA DEL DESARROLLO DEL PROYECTO DENTRO DE LA ASIGNATURA
%----------------------------------------------------------------------------------------------
Se presentan a continuación algunas ideas de los miembros del equipo para la posible mejora del desarrollo del proyecto dentro del ámbito de la asignatura de Proyecto Software:

\begin{itemize}
	\item Quizás sería interesante que se ahondara un poco más en el aprendizaje de utilización de la herramienta de control de versiones. Creémos no se le ha dado todo el uso que se lo podía dar, por miedo a destruir parte del proyecto (por ejemplo no hemos utilizado las ramificaciones), y estaría bien nos enseñaran a sacarle el máximo partido.
	
	\item De igual manera también se podría añadir algún apartado de enseñanza sobre herramientas útiles para la documentación, ya que es bastante lo que hay que documentar. Por ejemplo la herramienta LaTex.
	
	\item Algunos aspectos de ciertos roles todavía no nos han quedado claros, no hemos llegado a entenderlos totalmeente. Evidentemente el tamaño del proyecto no da para que los roles se desarrollen en su totalidad, y por ello ciertas cosas quedan un poco en el aire. Especialmente en relación al Gestor de Configuraciones y el Gestor de Calidad.
	
\end{itemize}

% 10.4
\subsection{Valoraciones subjetivas}
% 10.4. VALORACIONES SUBJETIVAS PERO ARGUMENTADAS
%----------------------------------------------------------------------------------------------
Como ya se ha dicho anteriormente, valoramos de forma bastante positiva el proyecto desarrolado en la asignatura, asi como la forma de trabajar que se nos ha impuesto/planteado para dicho desarrollo.

Se ha aprendido bastante en el aspecto de organización, planificación y trabajo en grupo; y no sólo aprendido, sino captado la importancia de ello en el desarrollo de un proyecto.

Personalmente, pensamos que hicimos una gran elección al decidir realizar un catálogo electrónico para la web alejándonos de la programación con \textit{Java GUI} o \textit{Android}, que seguramente hubieran complicado la implementación y nos hubiera quitado tiempo para el resto de apartados del proyecto.

Estamos además muy satisfechos con el rendimiento del equipo y con el hecho de que no hayan existido malos rollos ni problemas de esa índole, lo que ha facilitado mucho todo el desarrollo del proyecto. Valoramos también la ayuda del profesor Javier Zarazaga en las reuniones varias que se han realizado, y que con la dinámica que impone en ellas favorece ese buen trabajo en grupo que hemos tenido y el solucionar de la mejor manera posible todos los posibles problemas que iban surgiendo.

En resumen, una gran experiencia es la adquirida en este proyecto realizado y que estamos seguros en un futuro nos sirva como base para próximos proyectos en los que los diferentes miembros del equipo nos veamos involucrados.

%==============================================================================================
% ANEXOS
\appendix
%\addcontentsline{toc}{section}{ANEXOS}

%==============================================================================================
% A. CURRICULUMS VITAE
% A.1. CURRICULUM VITAE: DANIEL
% A.2. CURRICULUM VITAE: JAVIER
% A.3. CURRICULUM VITAE: JAVIER
% A.4. CURRICULUM VITAE: ALEJANDRO
% A.5. CURRICULUM VITAE: ALBERTO
% A.6. CURRICULUM VITAE: HÉCTOR
%==============================================================================================

% A
\section{Currículums Vitae}\label{sec:cvs}

% A.1
\subsection{CV: Daniel García Páez}
% A.1. CURRICULUM VITAE: DANIEL
%----------------------------------------------------------------------------------------------
\begin{cv}{}
\vspace{2em}	
\begin{cvlist}{Lenguajes de programación utilizados}
\item Conocimiento avanzado de Java
\item Conocimiento medio-avanzado de HTML,CSS,JSP,XML. Programación Web
\item Conocimiento deMySQL,SQL
\item Conocimiento medio de programación Android
\item Conocimiento medio de C
\item Conocimiento básico de PHP
\end{cvlist}

\begin{cvlist}{Experiencia}

	\item Realización de una aplicación web (HTML, CSS, JSP, XML) de red social.
	\item Realización de una aplicación web (HTML, CSS, JSP, XML) de operaciones con clientes y
cuentas bancarias.
	\item Creación y mantenimiento de varias bases de datos en MySQL.
	\item Realización de una aplicación móvil de gestión de notas para Android. Incluido todo el
proceso de análisis, requisitos, diseño e implementación del proyecto.

\end{cvlist}

\begin{cvlist}{Formación}

	\item[2008 a 2014] Estudiante de la EINA
		Grado en \textbf{Ingeniería informática} en la rama de \textbf{Software}\\
	\item[Inglés]  nivel avanzado(B2) de la escuela de Cambridge.


\end{cvlist}

\end{cv}
\newpage

% A.2
\subsection{CV: Javier Briz Alastrué}
% A.2. CURRICULUM VITAE: JAVIER
%----------------------------------------------------------------------------------------------
\begin{cv}{}
\vspace{2em}
\begin{cvlist}{Lenguajes de programación utilizados}
\item Java, C, C++, Wiring, Python, Haskell, Ada, Perl, Bash
\item HTML CSS,JSP, JavaScrip
\item MySQL
\end{cvlist}

\begin{cvlist}{Experiencia}

	\item[2013] Co-fundador de Prototyp3D y FaryNozzle. Experiencia en el ambito de la impresión 3D.
	\item[2007 - actualidad] Presidente en Púlsar, Certificado de servicios distinguidos de la Universidad.
	\item[2008 - actualidad] Secretario en ISC.
	\item[2008 - actualidad] Administrador de sistemas y clusters de computación en el grupo de Fluidodinámica Numérica de la Universidad de Zaragoza.
	\item[2008] Administrador de servidores y estaciones de trabajo en el Área de Mecánica de Fluidos de la Universidad de Zaragoza.

\end{cvlist}

\begin{cvlist}{Formación}

	\item[2007 - 2014] Estudiante de la EINA: Grado en \textbf{Ingeniería informática}
	\item[2011]: Certificado en "Software as a Service"(a course of study initiative of The University of California at Berkeley) 
	\item [2011] Curso de Administración de Sistemas Linux, Universidad de Barcelona
	\item [Inglés] Nivel avanzado (B2) en la Escuela Oficial de Idiomas.

\end{cvlist}

\end{cv}
\newpage

% A.3
\subsection{CV: Simón Ortego Parra}
\input{tex/a_3_cv_simon.tex}
\newpage

% A.4
\subsection{CV: Alejandro Gracia Mateo}
% A.4. CURRICULUM VITAE: ALEJANDRO
%----------------------------------------------------------------------------------------------
\begin{cv}{}
\vspace{2em}

\begin{cvlist}{Lenguajes de programación utilizados}
\item Java, C, C++,(ensamblador)
\item ARM, ARM thumb
\item CLIPS, Haskell, erlang
\item HTML CSS,JSP, XML
\item Matlab,SQL
\end{cvlist}

\begin{cvlist}{Experiencia}

	\item[2013] Realización de un compresor de ficheros de texto.
	\item[2013] Realización de una aplicación de subastas online.
	\item[2013] Realización de una página de recomendación de películas.
	\item[2013] Realización de un sistema de chat para múltiples usuarios con Erlang.
	\item[2013] Realización de un compilador para un lenguaje similar a Pascal.
	

\end{cvlist}

\begin{cvlist}{Formación}
	\item[2010 a 2014] Estudiante de la EINA
		Grado en \textbf{Ingeniería informática} en la rama de \textbf{computación}
\end{cvlist}

\end{cv}
\newpage

% A.5
\subsection{CV: Alberto Berbel Aznar}
\input{tex/a_5_cv_alberto.tex}
\newpage

% A.6
\subsection{CV: Héctor Francia Molinero}
% A.6. CURRICULUM VITAE: HÉCTOR
%----------------------------------------------------------------------------------------------
\begin{cv}{}
\vspace{2em}

\begin{cvlist}{Lenguajes de programación utilizados}
\item Ada, Java, C, C++, Android
\item Ensamblador, ARM, ARM thumb
\item CLIPS, Haskell, erlang
\item HTML, CSS, XML
\item Matlab, SQL
\end{cvlist}

\begin{cvlist}{Experiencia}

	\item[2011] Participación en la creación de una empresa de ocio.
	\item[2011] Creación de un programa para verificar documentos bien formados en XML con Flex y Bison
	\item[2012] Creación de un juego de dominó con programación concurrente.
	\item[2013] Realización de un compresor de ficheros de texto.
	\item[2013] Creación de una BD ``policial'' en MySQL.
	\item[2013] Realización de una aplicación móvil de gestión de notas para Android.
	\item[2013] Realización de una página de recomendación de películas.
	\item[2013] Realización de un sistema de chat para múltiples usuarios con Erlang y Java.
	\item[2013] Realización de un compilador para un lenguaje similar a MiniLang.
	\item[2013] Creación de una página Web usando el CMS Wordpress.

\end{cvlist}

\begin{cvlist}{Formación}

	\item[2008 a 2011] Estudiante de \textbf{Ingeniería Superior informática}
	\item[2011 a 2014] Estudiante de la EINA
		Grado en \textbf{Ingeniería informática} en la rama de \textbf{computación}


\end{cvlist}

\end{cv}
\newpage

%==============================================================================================
% B. HOJAS DE ESFUERZOS
% B.1. HOJA DE ESFUERZOS: DANIEL
% B.1. HOJA DE ESFUERZOS: DANIEL
% B.2. HOJA DE ESFUERZOS: JAVIER
% B.3. HOJA DE ESFUERZOS: SIMÓN
% B.4. HOJA DE ESFUERZOS: ALEJANDRO
% B.5. HOJA DE ESFUERZOS: ALBERTO
% B.6. HOJA DE ESFUERZOS: HÉCTOR
%==============================================================================================

% B
\section{Hojas de esfuerzos}\label{sec:hojas_esfuerzos}
%\input{tex/b_0_hojas_de_esfuerzos.tex}
%
% B.1
\subsection{Hoja de esfuerzos: Daniel García Páez}
%\input{tex/b_1_hoja_de_esfuerzos_daniel.tex}
En el \hojaesfuerzos{daniel_recopilacion_esfuerzos}{siguiente enlace} se puede acceder a la hoja de 
esfuerzos de Daniel García.

% B.2
\subsection{Hoja de esfuerzos: Javier Briz Alastrué}
%\input{tex/b_2_hoja_de_esfuerzos_javier.tex}
En el \hojaesfuerzos{javier_recopilacion_esfuerzos}{siguiente enlace} se puede acceder a la hoja de 
esfuerzos de Javier Briz.

% B.3
\subsection{Hoja de esfuerzos: Simón Ortego Parra}
%\input{tex/b_3_hoja_de_esfuerzos_simon.tex}
En el \hojaesfuerzos{simon_recopilacion_esfuerzos}{siguiente enlace} se puede acceder a la hoja de 
esfuerzos de Simón Ortego.

% B.4
\subsection{Hoja de esfuerzos: Alejandro Gracia Mateo}
%\input{tex/b_4_hoja_de_esfuerzos_alejandro.tex}
En el \hojaesfuerzos{alejandro_recopilacion_esfuerzos}{siguiente enlace} se puede acceder a la hoja de 
esfuerzos de Alejandro Gracia.

% B.5
\subsection{Hoja de esfuerzos: Alberto Berbel Aznar}
%\input{tex/b_5_hoja_de_esfuerzos_alberto.tex}
En el \hojaesfuerzos{alberto_recopilacion_esfuerzos}{siguiente enlace} se puede acceder a la hoja de 
esfuerzos de Alberto Berbel.

% B.6
\subsection{Hoja de esfuerzos: Héctor Francia Molinero}
%\input{tex/b_6_hoja_de_esfuerzos_hector.tex}
En el \hojaesfuerzos{hector_recopilacion_esfuerzos}{siguiente enlace} se puede acceder a la hoja de 
esfuerzos de Héctor Francia.

%==============================================================================================
% C. CRONOGRAMAS
% C.1. CRONOGRAMA INICIAL
% C.2. CRONOGRAMA FINAL
%==============================================================================================

% C
\section{Cronogramas}\label{sec:cronogramas}

% C.1
\subsection{Cronograma inicial}
%\input{tex/c_1_cronograma_inicial.tex}
\vspace{.2cm}
\begin{figure}[ht]
	\centerline{\includegraphics[scale=0.80]{img/cronograma_inicial}}\
	\caption{Cronograma inicial del proyecto}
	\label{fig:cronogramaInicial}
\end{figure}
\newpage

% C.2
\subsection{Cronograma final}
%\input{tex/c_2_cronograma_final.tex}
\vspace{.2cm}
\begin{figure}[ht]
	\centerline{\includegraphics[scale=0.75]{img/cronograma_final}}\
	\caption{Cronograma real del proyecto}
	\label{fig:cronogramaFinal}
\end{figure}

%==============================================================================================
% D. INFORMES DE PRUEBAS
%----------------------------------------------------------------------------------------------

% D
\section{Informes de pruebas}\label{sec:informe_pruebas}
% H. INFORMES DE PRUEBAS
%----------------------------------------------------------------------------------------------
\paragraph{}En el \informepruebas{usearch_pruebas_sistema}{siguiente enlace} se puede acceder al informe completo de pruebas de sistema.

\paragraph{}En el \informepruebas{resultados_w3c_validator}{siguiente enlace} se puede acceder al informe de las pruebas realizadas sobre la vista de la aplicación con el \textit{W3C Validator}.


\end{document}
%=-=-=-=-=-=-==-=-=-=-=-=-=%
%     F   I   N            %
%=-=-=-=-=-=-==-=-=-=-=-=-=%

