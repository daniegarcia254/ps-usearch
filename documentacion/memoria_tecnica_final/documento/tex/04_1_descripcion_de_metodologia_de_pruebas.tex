% 04.1. DESCRIPCIÓN DE LA METODOLOGÍA SEGUIDA PARA LA REALIZACIÓN DE LAS PRUEBAS DEL SISTEMA
%----------------------------------------------------------------------------------------------
Puesto que para la realización de las pruebas hemos utilizado nuestros conocimientos previos adquiridos en la aplicación de Validación y Verificación en otros proyectos, para no desviarnos de lo conocido, nuestra metodología de pruebas se aproxima a la metodología de pruebas utilizada en dichos proyectos anteriores: TMAP Next (Test Management Approach).
\\[6pt]
Los fundamentos de TMAP se basan en cuatro elementos esenciales:
\begin{itemize}
	\item \textbf{Proceso dirigido por el negocio.} La economía marca el esfuerzo a realizar, y cuáles son los riesgos prioritarios.
	\item \textbf{Un proceso de pruebas estructurado.} Nos guía a la hora de responder a las cuestiones típicas de \textit{qué/cuándo, cómo, con qué y quién} (ciclo de vida de pruebas).
	\item \textbf{Un kit de herramientas.} Se ofrece información práctica para establecer la infraestructura (\textit{con qué}), las técnicas (\textit{cómo}), y la organización (\textit{quién}).
	\item \textbf{Método completo y adaptable.} Flexibilidad para adaptar la metodología a distintas situaciones de desarrollo: nuevos desarrollos, mantenimiento, desarrollo propio o basado en software comercial, ...
\end{itemize}
\paragraph{} Pasos a dar en la gestión de pruebas dirigida por el negocio:
\begin{enumerate} 
	\item Identificar los objetivos de las pruebas
	\item Determinar los riesgos
	\item Determinar si una característica/parte se debe probar de forma detallada o ligera
	\item Estimar y planificar
	\item Elegir las técnicas de prueba y ejecutarlas 
	\item Informar sobre el progreso, calidad
\end{enumerate}
