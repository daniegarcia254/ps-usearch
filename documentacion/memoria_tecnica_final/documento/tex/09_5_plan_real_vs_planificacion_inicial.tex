% 09.5. PLAN REAL VS PLANIFICACIÓN INCIAL
%----------------------------------------------------------------------------------------------
\paragraph{}Siguiendo el análisis del punto anterior podemos decir de nuevo que en general la planificación inicial se asemeja bastante al plan real del proyecto:

\begin{itemize}
	\item Se han cumplido todos los requisitos que se impusieron al principio para el catálogo, sin grandes dificultades.
	
	\item La diferencia entre horas planificadas de esfuerzo y las reales es bastante aceptable, siendo importante que no se han sobrepasado sino que se han quedado algo cortas. Es decir, señal de equipo efectivo y de que el proyecto ha ido bien.
	
	\item Se han seguido dentro de lo posible las fechas para las tareas de cada iteración, como se puede ver en los en los cronogramas del anexo \ref{sec:cronogramas}. Ha habido pequeños desajustes, pero que no han afectado de manera grave en ningún momento a desarrollo del proyecto.
	
	\item No ha habido que lidiar prácticamente con ninguno de los riesgos que se catalogaron al principio del proyecto.
	
	\item Aunque no había una estimación inicial sobre ello, quizás sí que se han realizado más reuniones de las que uno pudiera pensar para un proyecto de tan poco tiempo. Pero siempre siendo aprovechadas.
\end{itemize}

\paragraph{}En resumen, repetimos nos consideramos satisfechos con el resultado final respecto a la planificación inicial en todos los aspectos, pues no ha habido una gran diferencia.
