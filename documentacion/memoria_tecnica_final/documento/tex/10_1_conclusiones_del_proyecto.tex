% 10.1. CONCLUSIONES DEL PROYECTO
%----------------------------------------------------------------------------------------------
Como conclusiones generales del proyecto todos los miembros del grupo concidimos en que:
\begin{itemize}
	\item  Ha sido una experiencia enriquecedora en el aspecto de aprender a trabajar en grupo, tener una organización y jerarquía elaborada, cada miembro con su papel y tareas bien definidas. No es fácil ni estamos acostumbrados a trabajar con este nivel de organización. Aprender a trabajar como equipo, siempre teniendo en cuenta la opinión de todos e intentar llegar a la mejor idea de forma consensuada. Además, por suerte podemos decir en nuestro caso que todos los miembros del equipo han rendido como se esperaba y no hemos tenido que lidiar con ese tipo de problemas. Aunque no es bueno acostumbrarse a ello pues en la vida real no será tan idílico seguramente.
	
	\item Se destaca la importancia de la planificación inicial del proyecto, especialmente en la planificación de tareas, fechas para cada tarea y reparto y asignación de tareas entre los diferentes miembros del equipo. Todo ello facilita el desarrollo del proyecto.
	
	\item También sacamos como conclusión que es importante monitorizar las horas de esfuerzo de cada miembro del equipo, así siempre se sabe quien trabaja más o menos, pudiendo repartir las tareas en las diferentes iteraciones del proyecto en función a esas horas.
\end{itemize}

En resumen, las conclusinoes del proyecto son positivas mayormente. El poder elegir al inicio el tipo de aplicación que deseabamos desarrollar ha permitido que la carga de trabajo fuese razonable durante el proyecto y ello ha ayudado a centrarnos más en la parte de organización, planificación, gestión, etc... que más interesaba en esta asignatura.
