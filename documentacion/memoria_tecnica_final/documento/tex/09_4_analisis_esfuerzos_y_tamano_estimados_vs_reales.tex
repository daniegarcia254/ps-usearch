% 09.4. ANÁLISIS DIFERENCIAS ESFUERZOS Y TAMAÑOS REALES DEL PROYECTO VS LOS ESTIMADOS
%----------------------------------------------------------------------------------------------
\paragraph{}Como se ha visto en el apartado \ref{sec:esfuerzos_reales} de este documento, en el desarrollo del proyecto en general el equipo ha realizado menos horas de esfuerzo de las estimadas en la planificación.

\paragraph{}Esto es debido principalmente a que el equipo ya contaba con experiencia anterior en otros proyectos de esta índole en el apartado de implementación y pruebas principalmente, con lo que al finalel trabajo se ha realizado quizás en estos apartados con mayor soltura de la esperada.

\paragraph{}Aún así, si nos fijamos en esas gráficas (~\cref{fig:6162} y ~\cref{fig:6164}) se puede comprobar que el total de horas reales empleadas se encuentra en un termino medio entre las estimadas y las estimadas con el factor de corrección aplicado. Es decir, que simplemente aplicando un factor de correción menor (igual fue demasiado alto por el miedo a quedarnos cortos y el desconocimiento de nunca haber realzado antes una planificación de este tipo) las horas estimadas y reales hubieran estado a un nivel muy similar. Por lo tanto, damos por satisfactorio la diferencia final entre dichas horas de esfuerzo y nos servirá para próximos proyectos.