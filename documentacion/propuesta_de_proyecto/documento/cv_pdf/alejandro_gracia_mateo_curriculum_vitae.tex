\documentclass[letterpaper,12pt]{article}

\usepackage{currvita}         % Cargamos el m�dulo para Curriculum Vitae
\usepackage[spanish]{babel}   % Definimos el idioma espa�ol
\usepackage[latin1]{inputenc} % Cargamos el modulo de codificacion de caracteres para que acepte tildes y e�es
\usepackage{fullpage}         % Disminuye los m�rgenes, cabe m�s en cada p�gina

\title{Curriculum Vitae}
\author{Alejandro Gracia Mateo }
%\date{\today}

\topmargin  -1cm   % Para reducir el margen superior e inferior
\textheight 26cm  % Alto hoja tama�o carta 27.81cm

\begin{document}

\setlength{\cvlabelwidth}{40mm}  % Modifica el espacio para las etiquetas de los listados

\begin{cv}{Curriculum Vitae}

\vspace{1cm}
\textbf{Nombre completo:} Alejandro Gracia Mateo
\vspace{1cm}

\begin{cvlist}{Lenguajes de programaci�n utilizados}
\item Java, C, C++,(ensamblador)
\item ARM, ARM thumb
\item CLIPS, Haskell, erlang
\item HTML CSS,JSP, XML
\item Matlab,SQL
\end{cvlist}

\begin{cvlist}{Experiencia}

	\item[2013] \textbf{Realizaci�n de un compresor de ficheros de texto.}\\

	\item[2013] \textbf{Realizaci�n de una aplicaci�n de subastas online.}\\

	\item[2013] \textbf{Realizaci�n de una p�gina de recomendaci�n de pel�culas.}\\
	
	\item[2013] \textbf{Realizaci�n de un sistema de chat para m�ltiples usuarios con erlang.}
	
	\item[2013] \textbf{Realizaci�n de un compilador para un lenguaje similar a Pascal}
	

\end{cvlist}

\begin{cvlist}{Formaci�n}

	\item[2010 a 2014] Estudiante de la EINA
		Grado en \textbf{Ingenier�a inform�tica} en la rama de \textbf{computaci�n}\\


\end{cvlist}

\end{cv}

\end{document}
