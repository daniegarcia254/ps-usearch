\documentclass[letterpaper,12pt]{article}

\usepackage{currvita}         % Cargamos el m�dulo para Curriculum Vitae
\usepackage[spanish]{babel}   % Definimos el idioma espa�ol
\usepackage[latin1]{inputenc} % Cargamos el modulo de codificacion de caracteres para que acepte tildes y e�es
\usepackage{fullpage}         % Disminuye los m�rgenes, cabe m�s en cada p�gina

\title{Curriculum Vitae}
\author{Javier Briz Alastru�}
%\date{\today}

\topmargin  -1cm   % Para reducir el margen superior e inferior
\textheight 26cm  % Alto hoja tama�o carta 27.81cm

\begin{document}

\setlength{\cvlabelwidth}{40mm}  % Modifica el espacio para las etiquetas de los listados

\begin{cv}{Curriculum Vitae}

\vspace{0.8cm}
\textbf{Nombre completo:} Javier Briz Alastru�
\vspace{0.8cm}

\begin{cvlist}{Lenguajes de programaci�n utilizados}
\item Java, C, C++
\item HTML CSS,JSP, JavaScrip
\item MySQL
\end{cvlist}

\begin{cvlist}{Experiencia}

	\item[2013] Co�fundador de Prototyp3D y FaryNozzle.
� \subitem Impresi�n 3D, construcci�n de impresoras y creaci�n de piezas para impresoras RepRap.
� \subitem Automatizaci�n de impresoras 3D RepRap.
� \subitem Administraci�n de sistemas y desarrollos centrados en Raspberry Pi.
� \subitem Colaboraci�n en el desarrollo de Octoprint: Software anfitri�n para impresi�n 3D.
\item[2007  �  actualidad]  Presidente  en  P�lsar.  P�lsar  es  la  asociaci�n  de  Software  Libre  de  la  Universidad  de  Zaragoza.
\subitem Certificado de servicios distinguidos de la Universidad.
\item[2008  � actualidad  ISC]  Secretario.  ISC  es  una  junior empresa. Es una  asociaci�n que tiene como  fin acercar el  mundo
de la empresa a estudiantes universitarios.
\item[2008  � actualidad]  Administrador  de  sistemas  y  clusters de computaci�n en el grupo de Fluidodin�mica Num�rica de
la Universidad de Zaragoza becado.
\item[2008]  Administrador  de  servidores  y  estaciones  de   trabajo  en   el  �rea  de  Mec�nica  de  Fluidos  de  la
Universidad  de  Zaragoza.  Becado  (2008  �  mayo  2012)  y  Personal   de  Administraci�n   y  Servicios  (junio  2012�febrero
2014).
	

\end{cvlist}

\begin{cvlist}{Formaci�n}

	\item[2007 a 2014] Estudiante de la EINA
		Grado en \textbf{Ingenier�a inform�tica}
	\item[2011]: Certificado en "Software as a Service"(a course of study initiative of The University of California at Berkeley) 
	\item [2011] Curso de Administraci�n de Sistemas Linux, Universidad de Barcelona
	\item [Ingl�s] Nivel avanzado (B2) en la Escuela Oficial de Idiomas "Fernando L�zaro Carreter" en Zaragoza, julio 2009.
Tres semanas en la Universidad de Hull, julio de 2007. Trinity College London Spoken English grade 7, julio 2007


\end{cvlist}

\end{cv}

\end{document}
