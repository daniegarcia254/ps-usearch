\documentclass[10pt,spanish]{article}

%----------------------------------------------------------------------------------------
% Codificación, y usar una fuente similar a la Palatino (y no Latin Modern Roman)
\usepackage[utf8]{inputenc} % Acentos, etc.
\usepackage[T1]{fontenc}
\usepackage[spanish]{babel} % Castellano
\usepackage{tgpagella}      % Fuente similar a Palatino

%----------------------------------------------------------------------------------------
% MÁRGENES (menores que en otro tipo de documentos)
\addtolength{\oddsidemargin}{-.375in}
\addtolength{\evensidemargin}{-.375in}
\addtolength{\textwidth}{1.25in}

%\addtolength{\topmargin}{-.375in}
%\addtolength{\textheight}{.5in}

%----------------------------------------------------------------------------------------
% COLORES
\usepackage{xcolor,colortbl} % Colores personalizados y color fondo tablas
\definecolor{grisCabeceraTabla}{RGB}{220,220,220}
\definecolor{grisHeader}{RGB}{180,180,180}
	
%----------------------------------------------------------------------------------------
% TABLAS	  
%\usepackage{supertabular} % Tablas multipágina
\linespread {1.4} % Tamaño que ocupa una línea (celda)

\usepackage{array}
\usepackage{longtable}
\setlength\LTleft{0pt}
\setlength\LTright{0pt}
\setlength{\columnsep}{0em}

% Definir los estilos de las celdas para indentar el texto
\newcolumntype{L}[1]{>{\raggedright\let\newline\\\arraybackslash\hspace{0pt}}m{#1}}
\newcolumntype{C}[1]{>{\centering\let\newline\\\arraybackslash\hspace{0pt}}m{#1}}
\newcolumntype{R}[1]{>{\raggedleft\let\newline\\\arraybackslash\hspace{0pt}}m{#1}}

% Filas con columnas multiples
\newcommand{\mc}[2]{\multicolumn{#1}{|C{\dimexpr 1\linewidth-2\tabcolsep}|}{#2}}

% Colores de lineas de las tablas
%\arrayrulecolor{grisCabeceraTabla}
\arrayrulecolor{grisHeader}

% Eliminar espacio previo en lista de ítems
\usepackage{enumitem}
\setlist{nolistsep}

% Comandos especiales para crear items dentro de tabla y poder hacer
% una tabla de multiples paginas y que se pueda realizar el salto de
% pagina en cualquier punto.
\usepackage{scrextend}
\newenvironment{lvlOneItem}
	{\begin{addmargin}[2.5em]{1em}
	\vspace{-1em}
	\hspace{-1em}$\bullet$\hspace{0.5em}}
	{\vspace{-2em}\end{addmargin}}
\newenvironment{lvlTwoItem}
	{\begin{addmargin}[4em]{1em}
	\vspace{-1em}
	\hspace{-1em}$\circ$\hspace{0.5em}}
	{\vspace{-2em}\end{addmargin}}


\newcommand{\itemNvlUno}[1]{\begin{lvlOneItem}{#1}\end{lvlOneItem}\\}
\newcommand{\itemNvlDos}[1]{\begin{lvlTwoItem}{#1}\end{lvlTwoItem}\\}
\newcommand{\espacioSubtabla}{\\[0.5ex]}

%----------------------------------------------------------------------------------------
% METADATOS DEL PDF Y PDF CLICKEABLE
\usepackage{hyperref}
\usepackage{hyperxmp}

% DATOS A CAMBIAR
\newcommand{\numeroDeReunion}{04}
\newcommand{\tituloReunion}{\bf Seguimiento de la  propuesta de proyecto y 1ª iteración}
\newcommand{\nombreDelProyecto}{$\mu$Search}

\hypersetup{
	pdfauthor={Alberto Berbel Aznar, 
				Javier Briz Alastrué, 
				Héctor Francia Molinero, 
				Daniel García Páez,
				Alejandro Gracia Mateo,
				Simón Ortego Parra},
	pdftitle={E20 - Acta R\numeroDeReunion: $\mu$Search\tituloReunion},
	pdfsubject={Proyecto Software. Grado Ing. Informática. EINA. Unizar},
	pdfkeywords={},
	pdfcopyright={Copyright (C) 2014 by Alberto Berbel Aznar, 
				Javier Briz Alastrué, 
				Héctor Francia Molinero, 
				Daniel García Páez,
				Alejandro Gracia Mateo,
				Simón Ortego Parra. All rights reserved.},
	pdfproducer={PDFLatex},
	pdfcreator={ps2pdf},
	colorlinks=false
}

%----------------------------------------------------------------------------------------
% Encabezados y pies de pagina : FancyHdr
\usepackage{fancyhdr}
\pagestyle{fancy}
\fancyhf{} % borrar todos los ajustes
\setlength{\headheight}{15pt}
\usepackage{lastpage} % Para poner total de paginas en el footer, ej: pag 1/4

\fancyhead[L]{\color{grisHeader}{\large BITPARTY}}
\fancyhead[R]{\color{grisHeader}E20 - Acta R\numeroDeReunion \\ \nombreDelProyecto}
\fancyfoot[R]{\color{grisHeader}\thepage/\pageref*{LastPage}}

% Modifica el ancho de las líneas de cabecera y pie
\renewcommand{\headrulewidth}{0pt}
\renewcommand{\footrulewidth}{0pt}
\renewcommand{\headsep}{0.6in}
\renewcommand{\headwidth}{6in}

%----------------------------------------------------------------------------------------
%----------------------------------------------------------------------------------------
% INICIO DEL DOCUMENTO
%----------------------------------------------------------------------------------------

\begin{document}
	
\begin{center}	
\Large{Acta de Reunión Nº \numeroDeReunion
\newline\tituloReunion}
\end{center}
\vspace{1.5em}

% PRIMERA TABLA: INFORMACIÓN BÁSICA
\begin{longtable}{ | L{\dimexpr 0.420\linewidth-2\tabcolsep} |
				     L{\dimexpr 0.570\linewidth-2\tabcolsep} | }
\hline % ------------------------------------------------------------------------
\rowcolor{grisCabeceraTabla}
\mc{2}{\bf Información básica}  \\
%\hline % ------------------------------------------------------------------------
%{\bf Cliente} & Nombre del cliente (por definir o no es necesario ?)  \\
\hline % ------------------------------------------------------------------------
{\bf Proyecto} & $\mu$Search \\
\hline % ------------------------------------------------------------------------
{\bf Fecha y hora de reunión} & 06/03/14 - 10:00 \\
\hline % ------------------------------------------------------------------------
{\bf Lugar} & Seminario 25, Edif. Ada Byron. EINA. Unizar \\
\hline % ------------------------------------------------------------------------
{\bf Tipo de reunión} & Reunión TP6 con Profesor \\
\hline % ------------------------------------------------------------------------
\end{longtable}


%----------------------------------------------------------------------------------------
% SEGUNDA TABLA - ASISTENTES
\begin{longtable}{ | C{\dimexpr 0.070\linewidth-2\tabcolsep} |
                     L{\dimexpr 0.350\linewidth-2\tabcolsep} |
                     C{\dimexpr 0.370\linewidth-2\tabcolsep} |
                     C{\dimexpr 0.200\linewidth-2\tabcolsep} | }
\hline % ------------------------------------------------------------------------
\rowcolor{grisCabeceraTabla}
\mc{4}{\bf Asistentes} \\
\hline % ------------------------------------------------------------------------
{\bf Nº} & {\bf Nombre y Apellidos} & {\bf Cargo} & {\bf Rol} \\
\hline % ------------------------------------------------------------------------
{\bf 1} & Alberto Berbel Aznar & Verificación y validación & --  \\
\hline % ------------------------------------------------------------------------
{\bf 2} & Javier Briz Alastrué & Gestor de configuraciones & --  \\
\hline % ------------------------------------------------------------------------
{\bf 3} & Héctor Francia Molinero & Gestor de calidad & --  \\
\hline % ----------------------------------------------------
{\bf 4} & Daniel García Páez & Director del proyecto & -- \\
\hline % ------------------------------------------------------------------------
{\bf 5} & Alejandro Gracia Mateo & Gestor de planificación & --  \\
\hline % ------------------------------------------------------------------------
{\bf 6} & Simón Ortego Parra & Gestor de desarrollo & Secretario  \\
\hline % ------------------------------------------------------------------------
{\bf 7} & Fco. Javier Zaragaza Soria & -- & Preparador  \\
\hline % ------------------------------------------------------------------------
\end{longtable}


%----------------------------------------------------------------------------------------
% TERCERA TABLA - AUSENTES
%\begin{longtable}{ | C{\dimexpr 0.070\linewidth-2\tabcolsep} |
%                     L{\dimexpr 0.350\linewidth-2\tabcolsep}  |
%                     C{\dimexpr 0.570\linewidth-2\tabcolsep} | }
%\hline % ------------------------------------------------------------------------
%\rowcolor{grisCabeceraTabla}
%\mc{3}{\bf Ausentes} \\ 
%\hline % ------------------------------------------------------------------------
%{\bf Nº} & {\bf Nombre y Apellidos} & {\bf Cargo} \\
%\hline % ------------------------------------------------------------------------
%{\bf 1} & Alberto Berbel Aznar & Verificación y validación \\
%\hline % ------------------------------------------------------------------------
%{\bf 1} & Javier Briz Alastrué & Gestor de configuraciones \\
%\hline % ----------------------------------------------------
%{\bf 2} & Héctor Francia Molinero & Gestor de calidad \\
%\hline % ------------------------------------------------------------------------
%{\bf 2} & Daniel García Páez & Director del proyecto \\
%\hline % ------------------------------------------------------------------------
%{\bf 3} & Alejandro Gracia Mateo & Gestor de planificación \\
%\hline % ------------------------------------------------------------------------
%{\bf 4} & Simón Ortego Parra & Gestor de desarrollo \\
%\hline % ------------------------------------------------------------------------
%\end{longtable}


%----------------------------------------------------------------------------------------
% CUARTA TABLA - OBJETIVOS, CUERPO DE LA REUNIÓN, DECISIONES TOMADAS, ...
\begin{longtable}{ | C{\dimexpr\linewidth-2\tabcolsep} | }
\hline % ------------------------------------------------------------------------
\rowcolor{grisCabeceraTabla}
{\bf Objetivos} \\
\hline % ------------------------------------------------------------------------
\endfirsthead
\hline % ------------------------------------------------------------------------
\endhead
\espacioSubtabla
\hline % ------------------------------------------------------------------------
\endfoot
\hline % ------------------------------------------------------------------------
\endlastfoot

\itemNvlUno{El objetivo principal de está reunion es el de realizar un seguimiento, con el profesor preparador Fco. Javier Zaragza, de nuestra propuesta de proyecto y de como se está abordando la primera iteración del proyecto software a realizar para su lanzamiento para poder ir cerrando y concretando varios apartados.}\\

% ------------------------------------------------------------------------
\rowcolor{grisCabeceraTabla}
{\bf Cuerpo de la reunión} \\
\hline % ------------------------------------------------------------------------

\itemNvlUno{Cada uno de los asistentes se presentó a la hora.}
\itemNvlUno{El equipo pusó al día al preparador sobre el avance en la propuesta de proyecto, todo aquello que ya estaba decidido. }
\itemNvlUno{El preparador nos ofreció varios consejos y pautas para la modificación y mejora de los requisitos que habíamos establecido en reuniones previas.}
\itemNvlUno{Se pasó a discutir ampliamente con el preparador para recibir su consejo sobre dos de los aspectos más complicados de abordar de la propuesta de proyecto:}
\itemNvlDos{Planificación de la 1ª iteración del proyecto en tareas y asignación de horas aproximadas a cada tarea. Que diferentes técnicas se pueden seguir para realizar la aproximación.}
\itemNvlDos{Realización de la oferta de presupuesto del proyecto.}
\itemNvlUno{Por último, se negoció la fijación de fechas para:}
\itemNvlDos{Siguiente reunión de seguimiento de la primera iteración.}
\itemNvlDos{Lanzamiento de la segunda iteración.}
\\

\hline % ------------------------------------------------------------------------
\rowcolor{grisCabeceraTabla}
{\bf Decisiones tomadas}  \\
\hline % ------------------------------------------------------------------------

\itemNvlUno{Se decidió que había que mejorar y detallar ciertos aspectos de los requisitos del sistema:}
\itemNvlDos{La definición de elemento: microcontrolador caracterizado por nº de referencia, arquitectura, flash, memoria, ...}
\itemNvlDos{Establecer en base a que característica de los microcontroladores se mostrarán ordenados los listados de búsqueda.}
\itemNvlDos{Establecer los datos personales que el cliente deberá rellenar a la hora de pedir la generación de una factura.}
\itemNvlDos{En caso de existir alguna restricción o limitación sobre el sistema o plataforma de desarrollo del proyecto software, añadirlas como requisitos no funcionales.}
\itemNvlUno{Se decide que la división de tareas tanto para la planificación como para el presupuesto se realizará en base a los requisitos establecidos.}
\itemNvlUno{Se incluirá una primera actividad o tarea denominada como "Tarea 0" que será el lanzamiento del proyecto, incluyendo la estructura inicial del proyecto: zona de arranque, instalación del SGBD, creación de la BD...}
\itemNvlUno{Se decidió la no utilización de máquinas virtuales para trabajar en el mismo entorno. Cada miembro trabajará en su propio equipo con las versiones de las tecnologías (PHP, MySQL, HTML...) a utilizar ya establecidas.}
\itemNvlUno{Se decide modificar la hoja de esfuerzos proporcionada en clase, de forma que todas las tareas que cada miembro del equipo haga irán incluida en una de la siguientes tareas:}
\itemNvlDos{Tareas ligadas a los requisitos.}
\itemNvlDos{Documentación.}
\itemNvlDos{Pruebas.}
\itemNvlDos{Reuniones técnicas y no técnicas (o de planificación).}
\\

\hline % ------------------------------------------------------------------------
\rowcolor{grisCabeceraTabla}
{\bf Temas pendientes} \\
\hline % ------------------------------------------------------------------------

\itemNvlUno{Establecimiento de las fechas de:}
\itemNvlDos{Reunión de seguimiento de la primera iteración.}
\itemNvlDos{Fecha de lanzamiento de la segunda iteración.}
\itemNvlUno{Modificación final de los requisitos del sistema para la propuesta de proyecto.}
\itemNvlUno{Especificación de las tareas en las que se dividirá la planificación del proyecto y realización de la oferta de presupuesto a partir de las misma.}
\\

\hline % ------------------------------------------------------------------------
\rowcolor{grisCabeceraTabla}
{\bf Próxima reunión prevista} \\
\hline % ------------------------------------------------------------------------
Reunión semanal  12/03/2014 20:00
\end{longtable}


\end{document}