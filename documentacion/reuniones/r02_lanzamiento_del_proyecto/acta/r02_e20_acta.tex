\documentclass[10pt,spanish]{article}

%----------------------------------------------------------------------------------------
% Codificación, y usar una fuente similar a la Palatino (y no Latin Modern Roman)
\usepackage[utf8]{inputenc} % Acentos, etc.
\usepackage[T1]{fontenc}
\usepackage[spanish]{babel} % Castellano
\usepackage{tgpagella}      % Fuente similar a Palatino

%----------------------------------------------------------------------------------------
% MÁRGENES (menores que en otro tipo de documentos)
\addtolength{\oddsidemargin}{-.375in}
\addtolength{\evensidemargin}{-.375in}
\addtolength{\textwidth}{1.25in}

%\addtolength{\topmargin}{-.375in}
%\addtolength{\textheight}{.5in}

%----------------------------------------------------------------------------------------
% COLORES
\usepackage{xcolor,colortbl} % Colores personalizados y color fondo tablas
\definecolor{grisCabeceraTabla}{RGB}{220,220,220}
\definecolor{grisHeader}{RGB}{180,180,180}
	
%----------------------------------------------------------------------------------------
% TABLAS	  
%\usepackage{supertabular} % Tablas multipágina
\linespread {1.4} % Tamaño que ocupa una línea (celda)

\usepackage{array}
\usepackage{longtable}
\setlength\LTleft{0pt}
\setlength\LTright{0pt}
\setlength{\columnsep}{0em}

% Definir los estilos de las celdas para indentar el texto
\newcolumntype{L}[1]{>{\raggedright\let\newline\\\arraybackslash\hspace{0pt}}m{#1}}
\newcolumntype{C}[1]{>{\centering\let\newline\\\arraybackslash\hspace{0pt}}m{#1}}
\newcolumntype{R}[1]{>{\raggedleft\let\newline\\\arraybackslash\hspace{0pt}}m{#1}}

% Filas con columnas multiples
\newcommand{\mc}[2]{\multicolumn{#1}{|C{\dimexpr 1\linewidth-2\tabcolsep}|}{#2}}

% Colores de lineas de las tablas
%\arrayrulecolor{grisCabeceraTabla}
\arrayrulecolor{grisHeader}

% Eliminar espacio previo en lista de ítems
\usepackage{enumitem}
\setlist{nolistsep}

% Comandos especiales para crear items dentro de tabla y poder hacer
% una tabla de multiples paginas y que se pueda realizar el salto de
% pagina en cualquier punto.
\usepackage{scrextend}
\newenvironment{lvlOneItem}
	{\begin{addmargin}[2.5em]{1em}
	\vspace{-1em}
	\hspace{-1em}$\bullet$\hspace{0.5em}}
	{\vspace{-2em}\end{addmargin}}
\newenvironment{lvlTwoItem}
	{\begin{addmargin}[4em]{1em}
	\vspace{-1em}
	\hspace{-1em}$\circ$\hspace{0.5em}}
	{\vspace{-2em}\end{addmargin}}


\newcommand{\itemNvlUno}[1]{\begin{lvlOneItem}{#1}\end{lvlOneItem}\\}
\newcommand{\itemNvlDos}[1]{\begin{lvlTwoItem}{#1}\end{lvlTwoItem}\\}
\newcommand{\espacioSubtabla}{\\[0.5ex]}
\newcommand{\cabeceraTabla}[1]{\rowcolor{grisCabeceraTabla}{\bf #1}}

%----------------------------------------------------------------------------------------
% METADATOS DEL PDF Y PDF CLICKEABLE
\usepackage{hyperref}
\usepackage{hyperxmp}

% DATOS A CAMBIAR
\newcommand{\numeroDeReunion}{02}
\newcommand{\tituloReunion}{\bf Lanzamiento del proyecto}
\newcommand{\nombreDelProyecto}{$\mu$Search}

\hypersetup{
	pdfauthor={Alberto Berbel Aznar, 
				Javier Briz Alastrué, 
				Héctor Francia Molinero, 
				Daniel García Páez,
				Alejandro Gracia Mateo,
				Simón Ortego Parra},
	pdftitle={E20 - Acta R\numeroDeReunion: \tituloReunion},
	pdfsubject={Proyecto Software. Grado Ing. Informática. EINA. Unizar},
	pdfkeywords={},
	pdfcopyright={Copyright (C) 2014 by Alberto Berbel Aznar, 
				Javier Briz Alastrué, 
				Héctor Francia Molinero, 
				Daniel García Páez,
				Alejandro Gracia Mateo,
				Simón Ortego Parra. All rights reserved.},
	pdfproducer={PDFLatex},
	pdfcreator={ps2pdf},
	colorlinks=false
}

%----------------------------------------------------------------------------------------
% Encabezados y pies de pagina : FancyHdr
\usepackage{fancyhdr}
\pagestyle{fancy}
\fancyhf{} % borrar todos los ajustes
\setlength{\headheight}{15pt}
\usepackage{lastpage} % Para poner total de paginas en el footer, ej: pag 1/4

\fancyhead[L]{\color{grisHeader}{\large BITPARTY}}
\fancyhead[R]{\color{grisHeader}E20 - Acta R\numeroDeReunion \\ \nombreDelProyecto}
\fancyfoot[R]{\color{grisHeader}\thepage/\pageref*{LastPage}}

% Modifica el ancho de las líneas de cabecera y pie
\renewcommand{\headrulewidth}{0pt}
\renewcommand{\footrulewidth}{0pt}
\renewcommand{\headsep}{0.6in}
\renewcommand{\headwidth}{6in}

%----------------------------------------------------------------------------------------
%----------------------------------------------------------------------------------------
% INICIO DEL DOCUMENTO
%----------------------------------------------------------------------------------------

\begin{document}
	
\begin{center}	
\Large{Acta de Reunión Nº \numeroDeReunion\hspace{0.25em}-\hspace{0.25em}\tituloReunion}
\end{center}
\vspace{1.5em}

% PRIMERA TABLA: INFORMACIÓN BÁSICA
\begin{longtable}{ | L{\dimexpr 0.420\linewidth-2\tabcolsep} |
				     L{\dimexpr 0.570\linewidth-2\tabcolsep} | }
\hline % ------------------------------------------------------------------------
\rowcolor{grisCabeceraTabla}
\mc{2}{\bf Información básica}  \\
%\hline % ------------------------------------------------------------------------
%{\bf Cliente} & Nombre del cliente (por definir o no es necesario ?)  \\
\hline % ------------------------------------------------------------------------
{\bf Proyecto} & $\mu$Search \\ 
\hline % ------------------------------------------------------------------------
{\bf Fecha y hora de comienzo} & 20/02/14 - 12:00 \\
\hline % ------------------------------------------------------------------------
{\bf Lugar} & Seminario 21, Edif. Ada Byron. EINA. Unizar \\
\hline % ------------------------------------------------------------------------
{\bf Tipo de reunión} & Estándar \\
\hline % ------------------------------------------------------------------------
\end{longtable}


%----------------------------------------------------------------------------------------
% SEGUNDA TABLA - ASISTENTES
\begin{longtable}{ | C{\dimexpr 0.070\linewidth-2\tabcolsep} |
                     L{\dimexpr 0.350\linewidth-2\tabcolsep} |
                     C{\dimexpr 0.370\linewidth-2\tabcolsep} |
                     C{\dimexpr 0.200\linewidth-2\tabcolsep} | }
\hline % ------------------------------------------------------------------------
\rowcolor{grisCabeceraTabla}
\mc{4}{\bf Asistentes} \\
\hline % ------------------------------------------------------------------------
{\bf Nº} & {\bf Nombre y Apellidos} & {\bf Cargo} & {\bf Rol} \\
\hline % ------------------------------------------------------------------------
{\bf 1} & Alberto Berbel Aznar & Verificación y validación & --  \\
\hline % ------------------------------------------------------------------------
{\bf 2} & Javier Briz Alastrué & Gestor de configuraciones & --  \\
\hline % ------------------------------------------------------------------------
{\bf 3} & Héctor Francia Molinero & Gestor de calidad & --  \\
\hline % ----------------------------------------------------
{\bf 4} & Daniel García Páez & Director del proyecto & -- \\
\hline % ------------------------------------------------------------------------
{\bf 5} & Alejandro Gracia Mateo & Gestor de planificación & --  \\
\hline % ------------------------------------------------------------------------
{\bf 6} & Simón Ortego Parra & Gestor de desarrollo & Secretario  \\
\hline % ------------------------------------------------------------------------
\end{longtable}


%----------------------------------------------------------------------------------------
% TERCERA TABLA - AUSENTES
%\begin{longtable}{ | C{\dimexpr 0.070\linewidth-2\tabcolsep} |
%                     L{\dimexpr 0.350\linewidth-2\tabcolsep}  |
%                     C{\dimexpr 0.570\linewidth-2\tabcolsep} | }
%\hline % ------------------------------------------------------------------------
%\rowcolor{grisCabeceraTabla}
%\mc{3}{\bf Ausentes} \\ 
%\hline % ------------------------------------------------------------------------
%{\bf Nº} & {\bf Nombre y Apellidos} & {\bf Cargo} \\
%\hline % ------------------------------------------------------------------------
%{\bf 1} & Alberto Berbel Aznar & Verificación y validación \\
%\hline % ------------------------------------------------------------------------
%{\bf 1} & Javier Briz Alastrué & Gestor de configuraciones \\
%\hline % ----------------------------------------------------
%{\bf 2} & Héctor Francia Molinero & Gestor de calidad \\
%\hline % ------------------------------------------------------------------------
%{\bf 2} & Daniel García Páez & Director del proyecto \\
%\hline % ------------------------------------------------------------------------
%{\bf 3} & Alejandro Gracia Mateo & Gestor de planificación \\
%\hline % ------------------------------------------------------------------------
%{\bf 4} & Simón Ortego Parra & Gestor de desarrollo \\
%\hline % ------------------------------------------------------------------------
%\end{longtable}


%----------------------------------------------------------------------------------------
% CUARTA TABLA - OBJETIVOS, CUERPO DE LA REUNIÓN, DECISIONES TOMADAS, ...
\begin{longtable}{ | C{\dimexpr\linewidth-2\tabcolsep} | }
\hline % ------------------------------------------------------------------------
\cabeceraTabla{Objetivos} \\
\hline % ------------------------------------------------------------------------
\endfirsthead
\hline % ------------------------------------------------------------------------
\endhead
\espacioSubtabla
\hline % ------------------------------------------------------------------------
\endfoot
\hline % ------------------------------------------------------------------------
\endlastfoot

\itemNvlUno{Organizar de forma consensuada el comienzo del proyecto y algunos
aspectos generales.} \\

\hline % ------------------------------------------------------------------------
\cabeceraTabla{Cuerpo de la reunión} \\
\hline % ------------------------------------------------------------------------

\itemNvlUno{Cada uno de los asistentes se presentó a la hora.}
\itemNvlUno{Se analizaron los requisitos de la aplicación de nuevo y
se fijaron perfectamente todos los parámetros.}
\itemNvlUno{Se pasó a hablar de la arquitectura en alto nivel del sistema
a desarrollar y de realizar los diagramas explicativos apropiados (diagramas
de componentes y de despliegue).}
\itemNvlUno{Se establecieron las tecnologías y estándares de codificación del
proyecto software.}
\itemNvlUno{Se decidieron las herramientas usadas para la documentación durante 
el desarrrollo del proyecto.}
\itemNvlUno{Se establecieron los mecanismos de comunicación entre los componentes
del equipo.}
\itemNvlUno{Se realizó una planificación de las tareas ha realizar inmediatamente
para el desarrollo del documento de ``Propuesta de Proyecto''.}\\

\hline % ------------------------------------------------------------------------
\cabeceraTabla{Decisiones tomadas}  \\
\hline % ------------------------------------------------------------------------

\itemNvlUno{Se establecieron como tecnologías y estándares de codificación (los
establecidos por los desarrolladores de éstas) las siguientes:}
	\itemNvlDos{HTML 5.}
	\itemNvlDos{CSS 3.}
	\itemNvlDos{PHP 5 y CodeIgniter (la versión apropiada).}
	\itemNvlDos{MySQL.}
	\itemNvlDos{LaTex: para la generación de la órden de pedido automática.}
\itemNvlUno{Para la documentación durante el desarrrollo del proyecto
se van a utilizar las siguientes herramientas:}
	\itemNvlDos{LaTex: para todo tipo de documentación (Actas, Convocatorias,
	documentos formales,etc.)}
	\itemNvlDos{Excel o similar (.xls): para la anotación de los esfuerzos
	de cada uno de los miembros del equipo.}
	\itemNvlDos{Modelio: para realizar diferentes diagramas relacionados
	con la Ingeniería del Software.}
	\itemNvlDos{Subversion: para el almacenamiento de todos los documentos.}
\itemNvlUno{Para la comunicación entre los miembros se van a utilizar:}
	\itemNvlDos{Google Groups: para todo tipo de comunicación.}
	\itemNvlDos{Whatsapp: para la comunicación informal (problemas, retrasos, citas,...)}
\itemNvlUno{Se realizó la siguiente planificación de cara a la presentación de la
``Propuesta de Proyecto'' que incluye una aproximación del número de horas y
los miembros a los que se les asigna cada una de las tareas:}
	\itemNvlDos{[2 h.] - Diagrama E/R de la BD + prototipo inicial de la 
	aplicación en papel: Daniel y Javier.}
	\itemNvlDos{[1 h.] - Diseño arquitectural del sistema (diag. componentes 
	y despliegue): Alberto.}
	\itemNvlDos{[1 h.] - Plantillas de documentación en LaTeX para las actas,
	convocatorias y propuesta de proyecto: Simón.}
	\itemNvlDos{[1 h.] - Hojas de esfuerzos: Códigos de categorías (incluyendo un 
	documento de texto explicativo con código + nombre + descr.): Alejandro}
	\itemNvlDos{[2 h.] - Logotipo + Nombre de la empresa: Héctor.}\\

\hline % ------------------------------------------------------------------------
\cabeceraTabla{Temas pendientes} \\
\hline % ------------------------------------------------------------------------

\itemNvlUno{Elección del responsable de la propuesta de proyecto.}
\itemNvlUno{Calendario de trabajo.}
\itemNvlUno{Iniciar el uso de la Wiki + Subversion.} \\
	
\hline % ------------------------------------------------------------------------
\cabeceraTabla{Próxima reunión prevista} \\
\hline % ------------------------------------------------------------------------
Miércoles 26 de Febrero del 2014 a las 20:00 \\
\end{longtable}


\end{document}