\documentclass[10pt,spanish]{article}

%----------------------------------------------------------------------------------------
% Codificación, y usar una fuente similar a la Palatino (y no Latin Modern Roman)
\usepackage[utf8]{inputenc} % Acentos, etc.
\usepackage[T1]{fontenc}
\usepackage[spanish]{babel} % Castellano
\usepackage{tgpagella}      % Fuente similar a Palatino

%----------------------------------------------------------------------------------------
% MÁRGENES (menores que en otro tipo de documentos)
\addtolength{\oddsidemargin}{-.375in}
\addtolength{\evensidemargin}{-.375in}
\addtolength{\textwidth}{1.25in}

%\addtolength{\topmargin}{-.375in}
%\addtolength{\textheight}{.5in}

%----------------------------------------------------------------------------------------
% COLORES
\usepackage{xcolor,colortbl} % Colores personalizados y color fondo tablas
\definecolor{grisCabeceraTabla}{RGB}{220,220,220}
\definecolor{grisHeader}{RGB}{180,180,180}
	
%----------------------------------------------------------------------------------------
% TABLAS	  
%\usepackage{supertabular} % Tablas multipágina
\linespread {1.4} % Tamaño que ocupa una línea (celda)

\usepackage{array}
\usepackage{longtable}
\setlength\LTleft{0pt}
\setlength\LTright{0pt}
\setlength{\columnsep}{0em}

% Definir los estilos de las celdas para indentar el texto
\newcolumntype{L}[1]{>{\raggedright\let\newline\\\arraybackslash\hspace{0pt}}m{#1}}
\newcolumntype{C}[1]{>{\centering\let\newline\\\arraybackslash\hspace{0pt}}m{#1}}
\newcolumntype{R}[1]{>{\raggedleft\let\newline\\\arraybackslash\hspace{0pt}}m{#1}}

% Filas con columnas multiples
\newcommand{\mc}[2]{\multicolumn{#1}{|C{\dimexpr 1\linewidth-2\tabcolsep}|}{#2}}

% Colores de lineas de las tablas
%\arrayrulecolor{grisCabeceraTabla}
\arrayrulecolor{grisHeader}

% Eliminar espacio previo en lista de ítems
\usepackage{enumitem}
\setlist{nolistsep}

% Comandos especiales para crear items dentro de tabla y poder hacer
% una tabla de multiples paginas y que se pueda realizar el salto de
% pagina en cualquier punto.
\usepackage{scrextend}
\newenvironment{lvlOneItem}
	{\begin{addmargin}[2.5em]{1em}
	\vspace{-1em}
	\hspace{-1em}$\bullet$\hspace{0.5em}}
	{\vspace{-2em}\end{addmargin}}
\newenvironment{lvlTwoItem}
	{\begin{addmargin}[4em]{1em}
	\vspace{-1em}
	\hspace{-1em}$\circ$\hspace{0.5em}}
	{\vspace{-2em}\end{addmargin}}


\newcommand{\itemNvlUno}[1]{\begin{lvlOneItem}{#1}\end{lvlOneItem}\\}
\newcommand{\itemNvlDos}[1]{\begin{lvlTwoItem}{#1}\end{lvlTwoItem}\\}
\newcommand{\espacioSubtabla}{\\[0.5ex]}

%----------------------------------------------------------------------------------------
% METADATOS DEL PDF Y PDF CLICKEABLE
\usepackage{hyperref}
\usepackage{hyperxmp}

% DATOS A CAMBIAR
\newcommand{\numeroDeReunion}{05}
\newcommand{\tituloReunion}{\bf Práctica 3: Comienzo primera iteración}
\newcommand{\nombreDelProyecto}{$\mu$Search}

\hypersetup{
	pdfauthor={Alberto Berbel Aznar, 
				Javier Briz Alastrué, 
				Héctor Francia Molinero, 
				Daniel García Páez,
				Alejandro Gracia Mateo,
				Simón Ortego Parra},
	pdftitle={E20 - Acta R\numeroDeReunion: $\mu$Search\tituloReunion},
	pdfsubject={Proyecto Software. Grado Ing. Informática. EINA. Unizar},
	pdfkeywords={},
	pdfcopyright={Copyright (C) 2014 by Alberto Berbel Aznar, 
				Javier Briz Alastrué, 
				Héctor Francia Molinero, 
				Daniel García Páez,
				Alejandro Gracia Mateo,
				Simón Ortego Parra. All rights reserved.},
	pdfproducer={PDFLatex},
	pdfcreator={ps2pdf},
	colorlinks=false
}

%----------------------------------------------------------------------------------------
% Encabezados y pies de pagina : FancyHdr
\usepackage{fancyhdr}
\pagestyle{fancy}
\fancyhf{} % borrar todos los ajustes
\setlength{\headheight}{15pt}
\usepackage{lastpage} % Para poner total de paginas en el footer, ej: pag 1/4

\fancyhead[L]{\color{grisHeader}{\large BITPARTY}}
\fancyhead[R]{\color{grisHeader}E20 - Acta R\numeroDeReunion \\ \nombreDelProyecto}
\fancyfoot[R]{\color{grisHeader}\thepage/\pageref*{LastPage}}

% Modifica el ancho de las líneas de cabecera y pie
\renewcommand{\headrulewidth}{0pt}
\renewcommand{\footrulewidth}{0pt}
\renewcommand{\headsep}{0.6in}
\renewcommand{\headwidth}{6in}

%----------------------------------------------------------------------------------------
%----------------------------------------------------------------------------------------
% INICIO DEL DOCUMENTO
%----------------------------------------------------------------------------------------

\begin{document}
	
\begin{center}	
\Large{Acta de Reunión Nº \numeroDeReunion\hspace{0.25em}-\hspace{0.25em}\tituloReunion}
\end{center}
\vspace{1.5em}

% PRIMERA TABLA: INFORMACIÓN BÁSICA
\begin{longtable}{ | L{\dimexpr 0.420\linewidth-2\tabcolsep} |
				     L{\dimexpr 0.570\linewidth-2\tabcolsep} | }
\hline % ------------------------------------------------------------------------
\rowcolor{grisCabeceraTabla}
\mc{2}{\bf Información básica}  \\
%\hline % ------------------------------------------------------------------------
%{\bf Cliente} & Nombre del cliente (por definir o no es necesario ?)  \\
\hline % ------------------------------------------------------------------------
{\bf Proyecto} & $\mu$Search \\ 
\hline % ------------------------------------------------------------------------
{\bf Fecha y hora de comienzo} & 06/03/14 - 12:00 \\
\hline % ------------------------------------------------------------------------
{\bf Lugar} & Seminario 21, Edif. Ada Byron. EINA. Unizar \\
\hline % ------------------------------------------------------------------------
{\bf Tipo de reunión} & Estándar \\
\hline % ------------------------------------------------------------------------
\end{longtable}


%----------------------------------------------------------------------------------------
% SEGUNDA TABLA - ASISTENTES
\begin{longtable}{ | C{\dimexpr 0.070\linewidth-2\tabcolsep} |
                     L{\dimexpr 0.350\linewidth-2\tabcolsep} |
                     C{\dimexpr 0.370\linewidth-2\tabcolsep} |
                     C{\dimexpr 0.200\linewidth-2\tabcolsep} | }
\hline % ------------------------------------------------------------------------
\rowcolor{grisCabeceraTabla}
\mc{4}{\bf Asistentes} \\
\hline % ------------------------------------------------------------------------
{\bf Nº} & {\bf Nombre y Apellidos} & {\bf Cargo} & {\bf Rol} \\
\hline % ------------------------------------------------------------------------
{\bf 1} & Alberto Berbel Aznar & Verificación y validación & --  \\
\hline % ------------------------------------------------------------------------
{\bf 2} & Javier Briz Alastrué & Gestor de configuraciones & Cronometrador  \\
\hline % ------------------------------------------------------------------------
{\bf 3} & Héctor Francia Molinero & Gestor de calidad & Observador  \\
\hline % ----------------------------------------------------
{\bf 4} & Daniel García Páez & Director del proyecto & Director \\
\hline % ------------------------------------------------------------------------
{\bf 5} & Alejandro Gracia Mateo & Gestor de planificación & --  \\
\hline % ------------------------------------------------------------------------
{\bf 6} & Simón Ortego Parra & Gestor de desarrollo & Secretario  \\
\hline % ------------------------------------------------------------------------
\end{longtable}


%----------------------------------------------------------------------------------------
% TERCERA TABLA - AUSENTES
%\begin{longtable}{ | C{\dimexpr 0.070\linewidth-2\tabcolsep} |
%                     L{\dimexpr 0.350\linewidth-2\tabcolsep}  |
%                     C{\dimexpr 0.570\linewidth-2\tabcolsep} | }
%\hline % ------------------------------------------------------------------------
%\rowcolor{grisCabeceraTabla}
%\mc{3}{\bf Ausentes} \\ 
%\hline % ------------------------------------------------------------------------
%{\bf Nº} & {\bf Nombre y Apellidos} & {\bf Cargo} \\
%\hline % ------------------------------------------------------------------------
%{\bf 1} & Alberto Berbel Aznar & Verificación y validación \\
%\hline % ------------------------------------------------------------------------
%{\bf 1} & Javier Briz Alastrué & Gestor de configuraciones \\
%\hline % ----------------------------------------------------
%{\bf 2} & Héctor Francia Molinero & Gestor de calidad \\
%\hline % ------------------------------------------------------------------------
%{\bf 2} & Daniel García Páez & Director del proyecto \\
%\hline % ------------------------------------------------------------------------
%{\bf 3} & Alejandro Gracia Mateo & Gestor de planificación \\
%\hline % ------------------------------------------------------------------------
%{\bf 4} & Simón Ortego Parra & Gestor de desarrollo \\
%\hline % ------------------------------------------------------------------------
%\end{longtable}


%----------------------------------------------------------------------------------------
% CUARTA TABLA - OBJETIVOS, CUERPO DE LA REUNIÓN, DECISIONES TOMADAS, ...
\begin{longtable}{ | C{\dimexpr\linewidth-2\tabcolsep} | }
\hline % ------------------------------------------------------------------------
\rowcolor{grisCabeceraTabla}
{\bf Objetivos} \\
\hline % ------------------------------------------------------------------------
\endfirsthead
\hline % ------------------------------------------------------------------------
\endhead
\espacioSubtabla
\hline % ------------------------------------------------------------------------
\endfoot
\hline % ------------------------------------------------------------------------
\endlastfoot

\itemNvlUno{Postmortem de la propuesta de proyecto: analizar el funcionamiento del 
equipo durante la fase previa de realización de la propuesta de proyecto.}
\itemNvlUno{Lanzamiento de la primera iteración: organizar de forma consensuada la
 primera iteración del proyecto.}\\

\hline % ------------------------------------------------------------------------
\rowcolor{grisCabeceraTabla}
{\bf Cuerpo de la reunión} \\
\hline % ------------------------------------------------------------------------

\itemNvlUno{Cada uno de los asistentes se presentaron a la hora.}
\itemNvlUno{Cada uno de los miembros del equipo comentaron el número de horas que llevan trabajadas en el proyecto.}
\itemNvlUno{Se revisó la zona de trabajo compartido y se propusieron ideas para mejorar su gestión.}
\itemNvlUno{Se acordaron las tecnologías y herramientas necesarias para el desarrollo del proyecto.}
\itemNvlUno{Se refinaron los requisitos establecidos en reuniones anteriores y se definieron nuevos requisitos.}
\itemNvlUno{Se fijaron las tareas a abordar durante la primera iteración del proyecto.}
\itemNvlUno{Se estableció el orden en el que serán abordados los requisitos.}
\itemNvlUno{Se realizó una estimación conjunta de los esfuerzos de cada tarea.}
\itemNvlUno{Comentarios finales de las conclusiones obtenidas.}\\

\hline % ------------------------------------------------------------------------
\rowcolor{grisCabeceraTabla}
{\bf Decisiones tomadas}  \\
\hline % ------------------------------------------------------------------------

\itemNvlUno{Cada uno de los miembros del grupo tendrá relativa libertad para elegir el entorno
  de desarrollo siempre y cuando lo codificado sea un fichero de texto (UTF-8) y funcione en la máquina destinada a pruebas:} 
  \itemNvlDos{ \emph{MySQL} (mysql ver 14.14 Distrib 5.5.35, for 	debian-linux-gnu)}
  \itemNvlDos{\emph{Apache} (Apache/2.2.22 (Debian))}
  \itemNvlDos{\emph{PHP} (PHP 5.4.4-14+deb7u7)}
		  
\itemNvlUno{Es necesario realizar ''commits'' muy frecuentemente para evitar conflictos o minimizarlos.}
\itemNvlUno{Nueva codificación de los ficheros compartidos: todos los nombres en minúsculas y sin espacios, utilizar underscores para separar las palabras.}  
\itemNvlUno{Se utilizará la Wiki para todo tipo de documentación.}
\itemNvlUno{Los requisitos de la aplicación a desarrollar son:}
\itemNvlDos{Un microcontrolador (elemento) estará compuesto de los siguientes campos: Referencia única para cada elemento, Arquitectura, Frecuencia, Flash, RAM y precio.}
\itemNvlDos{Insertar un nuevo elemento en el carro de compra.}
\itemNvlDos{Eliminar un elemento del carro de compra.}
\itemNvlDos{Modificar la cantidad solicitada de un elemento del carro de compra.}
\itemNvlDos{Se podrá acceder a los elementos del catálogo mediante un listado en el que aparezcan todos los elementos que se encuentren en el catálogo.}
\itemNvlDos{Se permitirá realizar la búsqueda de productos en cada caso en función de un único campo de búsqueda.}
\itemNvlDos{Los resultados de la búsqueda se presentarán de la siguiente forma: Un listado (sin paginación) que muestre de cada elemento todos sus campos en columnas.}
\itemNvlDos{Se permitirá realizar pedidos que incluirán los datos del cliente cada vez. Es decir, no existirá persistencia de los datos del cliente tras realizar pedidos. Los pedidos contendrán la suficiente información para identificar a los clientes. Además, no permitirán la reserva de los productos solicitados, únicamente generarán un presupuesto del coste de los productos elegidos.}
\itemNvlDos{Los datos solicitados del cliente para los pedidos son los siguientes: Nombre, Apellidos, Dirección, Ciudad, Provincia, País, Código postal, Teléfono, Mail. Además CIF y Empresa aparecerán como campos opcionales que servirán de distinción entre particulares y entidades.}
\itemNvlDos{Se añadirá una vista diferente para la administración del catálogo a la que no podrán acceder los clientes y se ejecutará localmente.}
\itemNvlDos{La administración del catálogo permitirá insertar un elemento en el catálogo a partir de las arquitecturas disponibles.}
\itemNvlDos{La administración del catálogo permitirá eliminar un elemento del catálogo.}
\itemNvlDos{La administración del catálogo permitirá modificar un elemento del catálogo.} 

\itemNvlUno{Tareas que se abordarán en la primera iteración:}
\itemNvlDos{\emph{Tarea 0}: Instalación. Desarrollo de la aplicación base (diseño de MVC, objetos, plataforma), sin interfaz Web. Se trabajará con comandos que se invocarán desde PHP a la terminal.}
\itemNvlDos{\emph{Tarea 1}: Marco común de las páginas web.}
\itemNvlDos{\emph{Tarea 2}: Insertar, modificar y eliminar elementos.}
\itemNvlDos{\emph{Tarea 3}: Mostrar el listado completo de los elementos del catálogo.}
\itemNvlDos{\emph{Tarea 4}: Gestión de pedidos, primera versión sin generar PDF, en texto plano.}
\itemNvlDos{\emph{Tarea 5}: Gestión del carrito de compra.}
\itemNvlDos{La vista (interfaz web) se desarrollará en el momento conforme se vaya necesitando y quien lo necesite.}

\itemNvlUno{Estimación conjunta de esfuerzos (horas por persona trabajando en la tarea):}
\itemNvlDos{Tarea 0: 2 horas.}
\itemNvlDos{Tarea 1: 1-2 horas.}
\itemNvlDos{Tarea 2: 4 horas.}
\itemNvlDos{Tarea 3: 2 horas.}
\itemNvlDos{Tarea 4: 5 horas.}
\itemNvlDos{Tarea 5: 5 horas.}

\itemNvlUno{Los encargados de desarrollar la parte del modelo del sistema (patrón MVC) son: Javier y Simón.}
\itemNvlUno{Los encargados de desarrollar la parte de la vista del sistema (patrón MVC) son: Daniel, Alejandro, Héctor y Alberto.}

\itemNvlUno{El producto final será un servicio que incluirá: la página web, el servidor web (Apache), la base de datos, el sistema de gestión de la base de datos.}\\

\hline % ------------------------------------------------------------------------
\rowcolor{grisCabeceraTabla}
{\bf Temas pendientes} \\
\hline % ------------------------------------------------------------------------

\itemNvlUno{Finalización del presupuesto.}
\itemNvlUno{Finalización de planificación de tareas para el proyecto.}
\itemNvlUno{Finalización de la propuesta del proyecto.} \\

\hline % ------------------------------------------------------------------------
\rowcolor{grisCabeceraTabla}
{\bf Próxima reunión prevista} \\
\hline % ------------------------------------------------------------------------
Lunes 10 de Marzo de 2014 a las 20:00  \\
\end{longtable}


\end{document}